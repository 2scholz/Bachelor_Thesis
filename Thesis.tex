%% ----------------------------------------------------------------
%% Thesis.tex -- MAIN FILE (the one that you compile with LaTeX)
%% ---------------------------------------------------------------- 


% Set up the document
\documentclass[a4paper, 12pt, oneside]{Thesis}  % Use the "Thesis" style, based on the ECS Thesis style by Steve Gunn
\graphicspath{{Figures/}}  % Location of the graphics files (set up for graphics to be in PDF format)

% Include any extra LaTeX packages required
%\usepackage[square, authoryear, comma, sort&compress]{natbib}  % Use the "Natbib" style for the references in the Bibliography
%\usepackage{apalike}
\usepackage{pdfpages}
\usepackage{amsmath}
\usepackage{algorithm}
\usepackage{algpseudocode}
\usepackage{dirtree}
\usepackage{listings}
\usepackage{verbatim}  % Needed for the "comment" environment to make LaTeX comments
\usepackage{vector}  % Allows "\bvec{}" and "\buvec{}" for "blackboard" style bold vectors in maths
\hypersetup{urlcolor=black, colorlinks=true}  % Colours hyperlinks in blue, but this can be distracting if there are many links.
\usepackage[english]{babel}
\usepackage[square, authoryear]{natbib}
\usepackage[nonumberlist]{glossaries}	
\makeglossaries

\hyphenation{Wi-Fi}

\newglossaryentry{MonteCarlo}
{
    name=Monte Carlo localization,
    description={Robot self-localization using probability. Its basis is the recursive state estimation implemented with a particle filter}
}

\newglossaryentry{LaserRangeFinder}
{
    name=laser range finder,
    description={Sensor that uses laser beams to detect the distance of obstacles}
}

\newglossaryentry{GaussianProcess}
{
    name=Gaussian process,
    description={A collection of random variables, any
    finite number of which have a joint Gaussian distribution \citep[p. 13]{Rasmussen:2005:GPM:1162254}}
}

\newglossaryentry{RGBD}
{
    name=RGB-D sensor,
    description={A sensor that is able to sense colors and depth}
}

\newglossaryentry{BFGS}
{
    name=BFGS,
    description={An optimization algorithm, that can be used to find minima or maxima of functions. It uses both first order- and second order-derivative information, while not explicitly computing the second order-derivative}
}

\newglossaryentry{L-BFGS}
{
    name=L-BFGS,
    description={The limited-memory variant of the BFGS algorithm}
}

\newglossaryentry{GPS}{name={GPS},description={A global positioning system that is using satellites. It can be used to determine the current location with an error of a few meters}}

\newglossaryentry{LIDAR}{name={LIDAR},description={Short for ``Light Detection and Ranging''. It is a type of sensor, that is using light in order to measure the distance to obstacles in the environment. It can be used to create 2D or 3D representations of areas}}

\newglossaryentry{GlobalLocalization}{name={global localization},description={Global localization describes localization without any prior information about the current position. That means a robot would have to figure out its current position at the start of the localization}}

\newglossaryentry{MAC-address}{name={MAC-address},description={Short for media access control-address. This address is a commonly used identifier in networks. So for example a Wi-Fi receiver has its own MAC-address as does the access point. The MAC-address is unique for each interface}}

\newglossaryentry{Odometer}{name={odometer},description={An odometer is a common sensor used in order to measure the distance traveled}}

\newglossaryentry{RecursiveStateEstimation}{name={recursive state estimation},description={Using incoming data a current state is computed recursively, by using the previous state. The state is represented by a probability density function}}

\newglossaryentry{EnvironmentMeasurementData}{name={environment measurement data},description={Provided by sensors like laser range finders, cameras or sonars. The environment measurement data gives us information about the current state of the environment of a robot}}

\newglossaryentry{ControlData}{name={control data},description={Can be provided by the commands given to a robot, so for example its velocity. Alternatively often provided by sensors like odometers. Control data gives us information about the change of state of a robot}}

\newglossaryentry{Prior}{name={prior},description={A prior is also called prior probability distribution. It is a distribution that was created by making certain assumptions before taking further evidence into account. When the distribution is transformed by taking that further evidence into account, it is called posterior, or posterior distribution}}

\newglossaryentry{Posterior}{name={posterior},description={A posterior is also called posterior probability distribution. It is a distribution created from the prior distribution after taking further evidence into account that wasn't considered when creating the prior}}

\newglossaryentry{Particle}{name={particle},description={Particles are generally a small part of larger system. Here particles are used to refer to samples that represent parts of a state belief}}

\newglossaryentry{KidnappedRobotProblem}{name={kidnapped robot problem},description={Describes the problem when a robot is taken and brought to a different location while a localization is running. The problem is that the robot is now localized in the wrong position and thus the localization fails}}

\newglossaryentry{Hyperparameter}{name={hyperparameter},description={A hyperparameter is a parameter that can be learned. There is no need to set it manually}}

\newglossaryentry{SSID}{name={SSID},description={Short for service set identifier. It is used by Wi-Fi access points. The identifier is not unique. It is used to signal different access point that belong to the same Wi-Fi network}}

\newglossaryentry{Regression}{name={regression},description={Inferring the relationship between different variables from available scalars from said variables. In case of the example of the linear regression the result is a linear function. In the example of Gaussian process regression the result is a Gaussian process}}

\newglossaryentry{TrainingSet}{name={training set},description={A set of data that was recorded in order to discover the relationship between different variables. To discover those relationships regression can be used}}

\newglossaryentry{Kernel}{name={kernel},description={When used with Gaussian processes, kernels are also referred to as covariance function. They are used to measure the difference between two inputs. They can lift the inputs into a higher space, while the computations still happen in the input space. Thus one can save computation time by applying kernels. They can be used whenever an operation can be represented by a dot product}}

\newglossaryentry{LogLikelihood}{name={log-likelihood},description={It is the natural logarithm of a likelihood function. In some cases they are easier to work with than the original likelihood function. The logarithm is monotonically increasing and has the maximum at the same point as the likelihood function. Often the partial derivatives of the log-likelihood function is easier to compute than the partial derivatives of the likelihood function. Thus it is attractive to use in order to optimize hyperparameters by searching the maximum of a function}}

\newglossaryentry{GradientDescent}{name={gradient descent},description={An algorithm that uses first-order derivatives in order to find a local minimum of a function}}

\newglossaryentry{Rprop}{name={Rprop},description={Short for resilient backpropagation. Using first-order derivatives the algorithm can be used to find a local minimum. It only uses the sign of resulting partial derivatives when searching for the minimum}}

\newglossaryentry{Gyro}{name={gyro},description={In general gyro sensors are used to measure angular velocity. For mobile robots they are used to measure the change in orientation}}
%% ----------------------------------------------------------------
\begin{document}
\includepdf[pages=-, offset=75 -75]{./title/title.pdf}
\frontmatter	  % Begin Roman style (i, ii, iii, iv...) page numbering

%% ----------------------------------------------------------------

\setstretch{1.3}  % It is better to have smaller font and larger line spacing than the other way round

% Define the page headers using the FancyHdr package and set up for one-sided printing
\fancyhead{}  % Clears all page headers and footers
%\rhead{\thepage}  % Sets the right side header to show the page number
\cfoot{\thepage}
\lhead{}  % Clears the left side page header

\pagestyle{fancy}  % Finally, use the "fancy" page style to implement the FancyHdr headers

% The Abstract Page
\addtotoc{Abstract}  % Add the "Abstract" page entry to the Contents
\abstract{ %\addtocontents{toc}{\vspace{1em}}  % Add a gap in the Contents, for aesthetics
\textbf{English:}\\
Localization is an important problem in robotics, that is needed for many navigation tasks. A sensor that is often used for this problem is the 2D laser range finder. It usually results in a precise localization as long as the robot's pose is roughly known at the start. Is this not the case the problem is specifically called global localization. Especially in rooms with a symmetric layout or rooms without any unique characteristics, a global localization using only a 2D laser range finder fails in many cases. Our approach here is to use Wi-Fi signal strength to infer an approximate position estimation. In this thesis we use Gaussian process regression with pre-recorded signal strength data of the environment in order to realize a Wi-Fi position estimation. The Wi-Fi position estimation is used to give a Monte Carlo localization with a 2D laser range finder an approximate initial position at the start in order to improve the global localization in buildings. The result is a higher success rate for the global localization when using the Wi-Fi position estimation compared to the approach only using the laser range finder. 

\textbf{Deutsch:}\\
Die Standortbestimmung ist ein wichtiges Problem in der Robotik. Sie wird für eine Vielzahl von Navigationsaufgaben benötigt. 2D Laserentfernungsmesser sind beliebte Sensoren für diese Aufgabe. In den meisten Fällen führt dies zu einer präzisen Lokalisation, zumindest wenn die Position des Roboters beim Start bekannt ist. Wenn dies nicht der Fall ist, spricht man auch von einer globalen Lokalisation. Insbesondere in Räumen mit symmetrischem Strukturen oder Räumen ohne auffallende Eigenschaften, führt eine globale Lokalisation, die nur einen 2D Laserentfernungsmesser nutzt, zum Misserfolg. Unser Ansatz ist in diesen Fällen Wi-Fi-Signalstärken zu verwenden, um daraus eine Position zu approximieren. Mit Hilfe von Gaußprozessregression setzen wir eine Wi-Fi Positionsschätzung um. Hierfür werden vorher Signalstärken in der Umgebung aufgenommen. Diese Wi-Fi-Positionsschätzung wird dann zusammen mit einer Monte Carlo Lokalisation, die einen 2D Laserentfernungsmesser verwendet, benutzt um die globale Lokalisation in Gebäuden zu verbessern. Das Resultat ist eine höhere Erfolgsrate für die globale Lokalisation, im Vergleich zum Ansatz der nur den 2D Laserentfernungsmesser verwendet.

}

\clearpage  % Abstract ended, start a new page
%% ----------------------------------------------------------------

\setstretch{1.0}  % Reset the line-spacing to 1.3 for body text (if it has changed)

\clearpage  % End of the Acknowledgements
%% ----------------------------------------------------------------

\pagestyle{fancy}  %The page style headers have been "empty" all this time, now use the "fancy" headers as defined before to bring them back


%% ----------------------------------------------------------------
\lhead{\emph{Contents}}  % Set the left side page header to "Contents"
\tableofcontents  % Write out the Table of Contents

%% ----------------------------------------------------------------

\setstretch{1.5}  % Set the line spacing to 1.5, this makes the following tables easier to read
\clearpage  % Start a new page
\lhead{\emph{Abbreviations}}  % Set the left side page header to "Abbreviations"
\listofsymbols{ll}  % Include a list of Abbreviations (a table of two columns)
{
% \textbf{Acronym} & \textbf{W}hat (it) \textbf{S}tands \textbf{F}or \\
\textbf{AMCL} & \textbf{A}daptive \textbf{M}onte \textbf{C}arlo \textbf{L}ocalization \\
\textbf{BFGS} & \textbf{B}royden \textbf{F}letcher \textbf{G}oldfarb \textbf{S}hanno \\
\textbf{CSV} & \textbf{C}omma \textbf{S}eperated \textbf{V}alues \\
\textbf{GPS} & \textbf{G}lobal \textbf{P}ositioning \textbf{S}ystem \\
\textbf{L-BFGS} & \textbf{L}imited-memory \textbf{B}royden \textbf{F}letcher \textbf{G}oldfarb \textbf{S}hanno \\
\textbf{LIDAR} & \textbf{Li}ght \textbf{D}etection \textbf{A}nd \textbf{R}anging \\
\textbf{MAC-address} & \textbf{M}edia \textbf{A}ccess \textbf{C}ontrol-address \\
\textbf{RGB-D} & \textbf{R}ed \textbf{G}reen \textbf{B}lue-\textbf{D}epth \\
\textbf{ROS} & \textbf{R}obot \textbf{O}perating \textbf{S}ystem\\
\textbf{Rprop} & \textbf{R}esilient back\textbf{prop}agation \\
\textbf{SSID} & \textbf{S}ervice \textbf{S}et \textbf{Id}entifier \\
\textbf{Wi-Fi} & \textbf{Wi}reless \textbf{Fi}delity
}

%% ----------------------------------------------------------------
\clearpage
\lhead{\emph{Glossary}}  % Set the left side page header to "Abbreviations"
\printglossaries{
\addtotoc{Glossary}
\addtocontents{toc}{\vspace{1em}}}  % Add the "Abstract" page entry to the Contents}

%----------------------------------------------------------------
% Begin the Dedication page

\setstretch{1.0}  % Return the line spacing back to 1.3


%% ----------------------------------------------------------------
\mainmatter	  % Begin normal, numeric (1,2,3...) page numbering
\pagestyle{fancy}  % Return the page headers back to the "fancy" style

% Include the chapters of the thesis, as separate files
% Just uncomment the lines as you write the chapters

%\input{./Chapters/Chapter1} % Introduction
\chapter{Introduction} % Write in your own chapter title
\label{Chapter1}
\lhead{Chapter 1. \emph{Introduction}} % Write in your own chapter title to set the page header

\section{Motivation}
Localization is an important problem in robotics. It is needed in order to solve a wide range of problems. Even simple tasks in households require a robot to be able to determine its own position in the environment. It needs to be able to find from point A to point B. Being able to do this, a robot could for example bring a fresh cup of coffee and bring the used cup back to the kitchen. It can help with cleaning tasks.\cite{pinheiro2015cleaning} It can also be highly important in industrial environments. Hamburg has one of the world's largest and busiest ports. In order to bring the shipping containers from one place to the other autonomous transport vehicles can be used. These are just a few of many real world examples, where localization is used.

The Monte Carlo Localization\cite{Dellaert_1999_533} is one of the most widely used algorithms to solve the localization problem. It has proven itself to be stable and reliable in a wide range of situations. As \cite{DBLP:conf/rss/LevinsonMT07} showed it was even suitable for use in autonomous vehicles to enhance the location estimation precision using a LIDAR, GPS and odometry.
In the case of the autonomous vehicle it is reasonable to expect that GPS will be available most of the time. It can give the vehicle a good estimation even though it can be a few meters off. In the case of the autonomous car the localization happens on a huge map and the GPS estimation is very helpful for the Monte Carlo Localization. Monte Carlo Localization works best when the position at the beginning is already known or if it at least gets a rough estimate. There are approaches to localize the robot with no prior position estimation, for example by taking the whole map into account instead of just a certain position, in the beginning. This is called global localization.

There are many situations where a robot can't rely on GPS. In many buildings its signal is obstructed. Now one could use global localization in the beginning.
But this can cause problems. On many maps there just aren't enough unique landmarks to quickly figure out the robot's pose. So in many cases the robot is not able to localize itself correctly. Once a wrong pose was determined by the localization, it can be difficult to recover from such a failure.

There are some environments that heighten the chance of the occurrence of such problems. In a long hallway, for example, many parts of the outline look similar. Also they are often symmetric and that can cause additional problems, because suddenly parts of the map are indistinguishable from each other. Also many buildings have different rooms with the exact same layout. Another situation where the global localization could fail would be when the laser scanner doesn't reach any obstacles. The robot would need to drive around and find them in order to actually localize itself. 
 
In most buildings there are a lot of different Wi-Fi networks. The networks MAC-address provides a unique identifier for each signal. The signal strengths can be a good indicator about how far the wireless access point is away. And while one signal alone wouldn't be enough to infer the position multiple signals from different access points can be used to do just that.

This could be used to enhance Monte Carlo localization with laser scanners. When at the start no position estimation is given it could be computed with the Wi-Fi data. Also in case of a localization failure the Wi-Fi signals could be used to infer the correct position.

This combines the strengths of both the position estimation with Wi-Fi signals and the localization with a laser scanner. The Wi-Fi signals have a unique identifier and will only be observed in a certain range on a map. The outline of a wall could be nearly identical in different places, but overall the localization with a laser scanner is very accurate once the true position was determined. 

\section{Related Work}
As already mentioned one of the most popular methods for localization is the Monte Carlo localization. We are going to use this method, both for localization via a 2D range finder, and parts of it also to estimate the position via Wi-Fi. \cite{Thrun:2005:PR:1121596} give an extensive overview over the Monte Carlo localization.

In order for the Monte Carlo localization to work, one needs a measurement model for the sensor that is used. \cite{Thrun:2005:PR:1121596} provide models that can be used for a 2d range finder. In order to estimate the position with a Wi-Fi receiver, one needs to implement a different measurement model.

There have been different approaches to solving the localization problem with Wi-Fi signals. \cite{serrano2012robot} used a propagation model to estimate the signal strength at a given point on the map. The advantage of this technique is that only the positions of the access points have to be known. But it is a lot more inaccurate than other approaches. The problem is that inside a building the Wi-Fi signal is often obstructed by obstacles, like walls, furniture or humans.  This makes the propagation model very inaccurate and thus also leads to inaccurate position estimations.

A different approach to the problem uses a map that was created beforehand. \cite{biswas2010wifi} recorded Wi-Fi signals in grid like fashion, recording in a certain interval. In order to interpolate the Wi-Fi signal from the recordings, linear interpolation was used. The drawback here is that the Wi-Fi signals have to be recorded at certain positions on the map. This makes the process of creating the map more complicated and time consuming.

In order to create something akin to a map from recorded Wi-Fi data, using regression proved to be useful. As it turns out Gaussian processes are well suited for this kind of problem.\\ \cite{Rasmussen:2005:GPM:1162254} give an extensive overview of how they can be used in machine learning. 

\cite{ferris2006gaussian} use Gaussian processes to build the map. This means the signal strength can be recorded at random places. Thus the needed data for creating the map can be recorded while the robot drives around or even does other tasks. 
\cite{duvallet2008wifi} use this approach to estimate the position of industrial vehicles. Here they used the Wi-Fi position estimation as a seed for a localization system using laser scanners. So for example in case the system fails and the computed location is inaccurate it uses the Wi-Fi position estimation for a rough estimate and a new starting point for the localization system.
\cite{DBLP:conf/icra/ItoEKTSB14} had a similar approach where they used Wi-Fi data for initialization for a global Monte Carlo localization with a RGB-D sensor. Both \cite{DBLP:conf/icra/ItoEKTSB14} and \cite{duvallet2008wifi} use Gaussian processes to estimate a position via Wi-Fi. 

\chapter{Basics} % Write in your own chapter title
\label{Chapter2}
\fancyhead[LE,RO]{Chapter 2. \emph{Basics}}
%\lhead{Chapter 2. \emph{Basics}} % Write in your own chapter title to set the page header
This chapter summarizes the basics of robot self-localization. At first in section \ref{sec:localization} we will give an example of data produced by a laser range finder. Next we will show how uncertainty can complicate the task of localization and how probability can be helpful to approach this problem. We will introduce the \gls{RecursiveStateEstimation} and building on this move on to an implementation of it using a \gls{Particle} filter. Afterwards we will show how said \gls{Particle} filter can be used for robot self-localization with the \Gls{MonteCarlo}. In the next section we will explain the problem of \gls{GlobalLocalization} and then introduce the \gls{KidnappedRobotProblem} and a possible solution.

In section \ref{sec:wifi} we will look into the aspects of Wi-Fi that are important for the Wi-Fi position estimation.

Section \ref{sec:gp} will revolve around the Wi-Fi sensor model, that can be used with the \Gls{MonteCarlo}. We will define what a \Gls{GaussianProcess} is and derive it from a Bayesian treatment of the linear \gls{Regression}. 

\section{Localization} \label{sec:localization}
\subsection{Localization with a Laser Range Finder}
\begin{figure}[htbp]
	\centering
		\includegraphics[width=\textwidth,height=\textheight,keepaspectratio]{./Figures/turtlebotfront.JPG}
		\rule{35em}{0.5pt}
	\caption[TurtleBot]{The TurtleBot used for testing the software and carrying out the experiments in chapter \ref{Chapter4}.}
	\label{fig:turtlebot2}
\end{figure}
\begin{figure}[htbp]
	\centering
		\includegraphics[width=\textwidth,height=\textheight,keepaspectratio]{./Figures/lrf_localization.png}
		\rule{35em}{0.5pt}
	\caption[Example of localization with laser range finder]{Example of localization with a laser range finder. The black circle (upper right) is the robot. The robot is on an obstacle map that was created beforehand, so the other black lines signify obstacles. The red squares are the laser range finder data. As we can see the red squares mostly fit to the obstacles.}
	\label{fig:laser_range_finder}
\end{figure}
The goal of localization is that a mobile robot is able to determine its own position. For this purpose we use a map of the environment of the robot. In our case the robot used is a TurtleBot 2 (http://www.turtlebot.com, accessed on 02.11.2016) shown in figure \ref{fig:turtlebot2}. This particular robot uses wheels to move around and a Kinect camera \citep{ece21221} and a Hokuyo UTM-30LX 2D \gls{LaserRangeFinder} \citep{laser} in order get information about the state of its environment. The robot also has sensors that provide us with information about its movement. An odometer is used to measure the distance traveled and a \gls{Gyro} to measure the angular velocity. Now using the information provided by the various sensors the objective is to infer the location of the robot. 

The 2D \gls{LaserRangeFinder} is a popular choice for this task. It gives us the distance to obstructions in the robot's environment and produces reliable data. The prices for these kind of sensors are reasonable. A 2D \gls{LaserRangeFinder} is able to determine the distance to obstacles by using laser beams. The beams get reflected back to the \gls{LaserRangeFinder} by the surfaces of nearby obstructions. The time it takes for the beams to arrive at the \gls{LaserRangeFinder} is used to infer the distance. The further an object is away, the longer it takes for a beam to be reflected back. This kind of data is easy to interpret and easy to process, compared to data produced by similar sensors like cameras. In order to infer useful information from camera pictures we would need to filter them.

Figure \ref{fig:laser_range_finder} shows the kind of data a laser range finder produces. Here the black circle is the robot. The red squares show us the distance to obstacles in the robot's environment. The position of the robot on the map mirrors its position in the real world. This map is an obstacle map, so it shows where we would expect obstacles in the environment. The map is also divided into small grid cells, that can be either occupied, free or unknown. Here black represents occupied areas, light gray represents free areas and dark gray represents unknown areas. As one can see the laser range finder data mostly matches our expectations. The obstacles found by the laser range finder match the obstacles shown on the map, so it is likely that the robot is located on the correct position on the map. So sensors like the laser range finder can be used to check if the robot is localized in the correct position by comparing the produced data with the expectations based on the map. Sensors like \glspl{Odometer} and \glspl{Gyro} can be used to move the robot accordingly on the map. So when the robot drives a few meters forward, the same should happen on the map. 

As laid out the robot has different kinds of sensors that provide us with information. Using that data and a map of the environment our goal is to infer the position of the robot.  But for a variety of reasons we can not be 100\% certain about the data sensors produce.

\subsection{Uncertainty}
\begin{figure}[htbp]
	\centering
		\includegraphics[width=\textwidth,height=\textheight,keepaspectratio]{./Figures/laser_range_finder_uncertainty.png}
		\rule{35em}{0.5pt}
	\caption[Example of uncertainty when localizing with a laser range finder]{This is an example of how uncertainty plays a role in localization. The figure is very similar to figure \ref{fig:laser_range_finder}, but signified by the blue circle there are measurements created by the laser range finder that do not fit to the map. The reason in this case was simply that a person was standing there. The green circles show doors that were opened when the map was created, but are now either closed or in a different position.}
	\label{fig:laser_range_finder_uncertainty}
\end{figure}
Uncertainty plays a big role in robotics. By dealing with the real world this inherently introduces unpredictability. The robot's environment is dynamic and can change constantly. Even the robot itself changes its own environment by acting in it. See for example figure \ref{fig:laser_range_finder_uncertainty}. It is similar to figure \ref{fig:laser_range_finder}, but here measurements appear that don't fit the expectation we would have given the map the robot is located on. The robot is still located on the right position that mirrors its position in the real world. What changed is the environment. Someone was standing in the room and the data produced by the laser range finder mirrors that circumstance. A similar example would be when furniture is moved to a different place. Suddenly the data will not match in a particular place. But it does not even have to be a permanent change. In the green circles in figure \ref{fig:laser_range_finder_uncertainty} two doorways are highlighted. When the map was created the doors stood open, but now the one on the lower right is closed and the one in the upper left is in a different position. It would be impractical to create a new map whenever there are changes in the environment. 

Robots usually have a variety of sensors, they use to get information about the state of their environment. The sensors used in robotics have certain limitations. They can only provide an inexact interpretation of the world around them and this is something we need to keep in mind. When an \gls{Odometer} tracks a robots path it will never be exact. When a laser scanner measures the distance to an obstacle we have to expect that there will be an error margin. The sensors have to deal with physical limitations when trying to capture the state of their environment. Often they are subject to noise. In localization 2D maps are used very often. They are a model of the real world. But once again it is only an approximate model that strips away a lot of information from the real world and thus introduces more uncertainty \citep[p.\ 3-4]{Thrun:2005:PR:1121596}.

Probability can be a helpful tool to circumvent this problem. When localizing instead of trying to find the pose of the robot directly we test how likely it is that the robot is in different hypothesized poses. So, using all the information we get from the sensors we compute which poses are the most likely ones \citep[p.\ 5]{Thrun:2005:PR:1121596}. In the following section we will show an algorithm that uses probability to deal with the uncertainty in our sensor data.
\subsection{Recursive State Estimation}
\citet{Thrun:2005:PR:1121596} divide the data usually collected by robots into two different categories:
\begin{enumerate}
	\setlength\itemsep{0 em}
	\item \Gls{EnvironmentMeasurementData}, denoted $z_{t_1:t_2}$ for data from time $t_1$ to time $t_2$
	\item \Gls{ControlData}, denoted $u_{t_1:t_2}$ for data from time $t_1$ to time $t_2$
\end{enumerate}
The first one gives us information about the current state of the environment, provided by sensors like cameras, laser range scanners or Wi-Fi receivers. The second kind of data gives us information about the change of state. These are values that are given to the robot to execute some action, for example a velocity. Sensors like odometers, a sensor that measures the distance traveled, could be classified as producing environment measurement data, but in practice this kind of data is often used as \gls{ControlData} \citep[p.\ 22-23]{Thrun:2005:PR:1121596}.

The current time step is denoted as $t$, so the current state is denoted $x_t$. At this point we will rely on probability for the reasons already explained. The probability of state $x_t$ is the following: $p(x_t|x_{0:t-1},z_{1:t-1}, u_{1:t})$. Here all \gls{ControlData} up to time step $t$ is taken into account and all \gls{EnvironmentMeasurementData} up to the previous time step $t-1$ is taken into account. But if we assume that state $x_t$ is a sufficient summary of everything that happened before time step $t$, then we can omit most data: 
\begin{equation} \label{eq:statetransition}
p(x_t|x_{0:t-1},z_{1:t-1}, u_{1:t}) = p(x_t|x_{t-1}, u_t)
\end{equation}
The probability that a certain measurement was observed can be described by $p(z_t|x_{0:t},z_{1:t-1}, u_{1:t})$. Once again we can omit most data:
\begin{equation} \label{eq:measurementprob}
p(z_t|x_{0:t},z_{1:t-1}, u_{1:t}) = p(z_t|x_t)
\end{equation}
Equation \ref{eq:statetransition} is known as the state transition probability and equation \ref{eq:measurementprob} is known as the measurement probability. The state transition probability takes the \gls{ControlData} into account. As we will see, for localization this means it is used to model the movement of the robot. The measurement probability on the other hand takes the measurement data into account. For localization this means it is used to reflect the information the robot has about its environment at the moment \citep[p.\ 24-25]{Thrun:2005:PR:1121596}.

The state of a robot is represented by so called belief distributions. 
\begin{equation} \label{eq:prediction}
\overline{bel}(x_t) = p(x_t|z_{1:t-1}, u_{1:t})
\end{equation}
Equation \ref{eq:prediction} is often referred to as prediction. It doesn't take the last environment measurement into account. 
\begin{equation} \label{eq:posterior}
bel(x_t) = p(x_t|z_{1:t}, u_{1:t})
\end{equation}
The \gls{Posterior} from equation \ref{eq:posterior} on the other hand does take the last measurement into account \citep[p.\ 25-26]{Thrun:2005:PR:1121596}. So $\overline{bel}(x_t)$ can be seen as the prior, that is computed with incomplete information and ${bel}(x_t)$ as the corresponding posterior, that is created by including the last measurement data. As can be seen in algorithm \ref{bayes_filter} the posterior from equation \ref{eq:posterior} is computed by using the prediction from equation \ref{eq:prediction}.

Once again, we don't want to take all past data from all time steps into account. So we are going to simplify both equations for our purposes. $bel(x_t)$ can be simplified by using Bayes' rule and equation \ref{eq:measurementprob}.
\begin{equation}\label{bayes}
\begin{aligned}
p(x_t|z_{1:t}, u_{1:t}) &= \dfrac{p(z_t|x_t,z_{1:t-1},u_{1:t})p(x_t|z_{1:t-1},u_{1:t})}{p(z_t|z_{1:t-1},u_{1:t})}\\
&= \eta p(z_t|x_t,z_{1:t-1},u_{1:t})p(x_t|z_{1:t-1},u_{1:t})\\
bel(x_t) &= \eta p(z_t|x_t)\overline{bel}(x_t)
\end{aligned}
\end{equation}
Here $\eta$ is a normalizer, stemming from the fact that $p(z_t|z_{1:t-1},u_{1:t})$ is independent of the state $x_t$ and so no matter what value $x_t$ takes on in $p(x_t|z_{1:t},u_{1:t})$, it does not influence $\eta$.

$\overline{bel}(x_t)$ can be simplified for our purposes, using the theorem of total probability and equation \ref{eq:statetransition} \citep[p.\ 31-33]{Thrun:2005:PR:1121596}.
\begin{equation} \label{eq:overbel}
\begin{aligned}
p(x_t|z_{1:t-1},u_{1:t}) &= \int p(x_t|x_{t-1},z_{1:t-1},u_{1:t})p(x_{t-1}|z_{1:t-1},u_{1:t})dx_{t-1}\\
&= \int p(x_t|u_t,x_{t-1})p(x_{t-1}|z_{1:t-1},u_{1:t-1})dx_{t-1}\\
\overline{bel}(x_t) &= \int p(x_t|u_t,x_{t-1})bel(x_{t-1})dx_{t-1}
\end{aligned}
\end{equation}
In the second step of equation \ref{eq:overbel} we make use of the fact that $u_t$ is not needed to infer the probability of state $x_{t-1}$ and therefore simply omit it. 
\ref{eq:overbel} and \ref{bayes} are used in the Bayes filter algorithm.
\begin{algorithm}
\caption{Bayes\_filter \citep[p.\ 27]{Thrun:2005:PR:1121596}}
\label{bayes_filter}
\begin{algorithmic}[1]
\Procedure{Bayes\_filter}{$bel(x_{t-1}),u_t,z_t$}
\For{all $x_t$}
\State $\overline{bel}(x_t) = \int p(x_t|u_t,x_{t-1})bel(x_{t-1})dx_{t-1}$
\State $bel(x_t) = \eta p(z_t|x_t)\overline{bel}(x_t)$
\EndFor
\State \Return $bel(x_t)$
\EndProcedure
\end{algorithmic}
\end{algorithm}

The inputs of the algorithm are the distribution of belief $bel(x_{t-1})$ from the last time step, the new environment data $z_t$ and \gls{ControlData} $u_t$ from this time step $t$. In line 3 the belief gets updated with the \gls{ControlData}, to make sure that the change of state gets reflected in the distribution before we apply our knowledge from the environment data. In line 4 the belief gets updated with the measurement data too. Now all available data was used to create the new belief distribution. The belief distribution can be used to infer the state $x_t$. 

\begin{figure}[htbp]
	\centering
		\includegraphics[page=1,trim={6cm 6.5cm 3cm 5.7cm},clip]{./Figures/fig.pdf}
		\rule{35em}{0.5pt}
	\caption[Localization Example]{A simple example to explain how the Bayes filter works. In (a) the belief is evenly distributed. In (b) the robot observes the door and the measurement data reflects this and updates $bel(x)$ accordingly. In (c) the robot moves to the right and the beliefs structure shifts to the right as well, but the peaks become lower, because the \gls{ControlData} could be imprecise. In (d) the robot takes the new measurement data into account. The belief clearly spikes at the correct position. In (e) the robot moved again. The spike is still in the correct position, but it is lower \citep[p.\ 6]{Thrun:2005:PR:1121596}.}
	\label{fig:loc_example}
\end{figure}

Figure \ref{fig:loc_example} shows a simple example how the algorithm could be used for localization. It shows a robot that is able to detect whether it is standing in front of a door. It reflects the nature of the \gls{ControlData} and measurement data. Whenever the robot has moved and the \gls{ControlData} is used to update the belief, the belief still spikes in the correct position, but the spike becomes weaker. We have to take into account that the \gls{Odometer} could give us imprecise measurements. This is the reason why we use the measurement data as well. Even if we had the correct position in the beginning, when we only used \gls{ControlData}, after driving around for some time the localization will be off by a huge margin. The measurement data is there to correct this. If we look at both \gls{ControlData} and measurement data in a certain time interval the uncertainty induced by moving around is small enough that we can compensate for it by using the measurement data. 

The figure also provides a good example as to why we use probabilities. In figure \ref{fig:loc_example} (b) the robot senses a door. There are three different doors, but the robot can't be sure which door it is, so it simply applies the same probabilities to all three doors instead of randomly choosing one. Then it also can't be sure about the exact position in front of these doors because the data could be imprecise. So in front of each door the distribution has bell-shaped spikes to reflect this. 

Another good example why probabilities work so well can be observed in (c). The robot moved and of course the \gls{ControlData} is also imprecise. So the spikes move according to the \gls{ControlData}, but they get weaker and more spread out.

Now (d) shows a nice example how the data from the previous time steps get taken into account. The spikes move according to the \gls{ControlData}. Now the robot observes a door once again, just like in (b). But because the belief has only a spike in front of the correct door, the new belief has a high and distinct spike in front of the correct door after taking the measurement data into account.
\subsection{Particle Filter}
The \gls{Particle} filter is a non-parametric solution to implement the Bayes filter.  Here the \gls{Posterior} distribution $bel(x_t)$ is represented by finitely many samples \citep[p.\ 85]{Thrun:2005:PR:1121596}.

Specifically it is represented by a set of \gls{Particle}s that is denoted as:
\begin{equation}
X_t = x_t^{[1]},x_t^{[2]},...,x_t^{[M]}
\end{equation}

Each \gls{Particle} represents a possible state hypothesis at time $t$. $M$ is the number of \gls{Particle}s \citep[p.\ 96-97]{Thrun:2005:PR:1121596}.

\begin{algorithm}
\caption{Particle\_filter \citep[p.\ 98]{Thrun:2005:PR:1121596}}
\label{particle_filter}
\begin{algorithmic}[1]
\Procedure{Particle\_filter}{$X_{t-1},u_t,z_t$}
\State $\bar{X_t} = X_t = \emptyset$
\For{$m = 1$ to $M$}
\State sample $x_t^{[m]} \sim p(x_t|u_t,x_{t-1}^{[m]})$
\State $w_t^{[m]} = p(z_t|x_t^{[m]})$
\State $\bar{X_t} = \bar{X_t} + \langle x_t^{[m]},w_t^{[m]}\rangle$
\EndFor
\For{$m = 1$ to $M$}
\State draw $i$ with probability $\propto w_t^{[i]}$
\State add $x_t^{[i]}$ to $X_t$
\EndFor
\State \Return $X_t$
\EndProcedure
\end{algorithmic}
\end{algorithm}

The belief $bel(x_t)$ is approximated by \gls{Particle} set $X_t$. The probability for a state hypothesis to be included is ideally proportional to its Bayes filter \gls{Posterior} $bel(x_t)$ \citep[p.\ 98]{Thrun:2005:PR:1121596}.
\begin{equation} \label{eq:particle_prob}
x_t^{[m]} \sim p(x_t|z_{1:t},u_{1:t})
\end{equation}

Just like in the Bayes filter the \gls{Posterior} $X_t$ is computed recursively from the set $X_{t-1}$. 

On line 4 of algorithm \ref{particle_filter} the equivalent to $\overline{bel}(x_t)$ is computed. The state $x_t^{[m]}$ is generated based on \gls{Particle} $x_{t-1}^{[m]}$ and \gls{ControlData} $u_t$. Based on $p(x_t|u_t,x_{t-1}^{[m]})$ new \gls{Particle}s are sampled. This is done for every \gls{Particle} in the set $X_{t-1}$. In this step the most recent information from the \gls{ControlData} gets added.

Line 5 is then the equivalent to computing $bel(x_t)$. For the set of particles $X_t$ we compute the corresponding weights. Here $w_t^{[m]}$ is the weight of \gls{Particle} $m$ at time $t$. The higher the weight the more likely it is that the state of the \gls{Particle} is representative of the real state. 

On line 9 new \gls{Particle}s get drawn. This happens according to the weights $w_t$. The higher the weight, the higher the chance of the \gls{Particle} to be drawn. The \gls{Particle}s that were unlikely to represent the real state won't get drawn and only the ones that have a high weight survive.


% % % % % % % % % % % % % % % % % % % % % % % % % % % % % % % % % % % % % % % % % % % % % % % % % % % % % % % % % % % % % % % % % % %
% Still todo: There is more stuff on the particle filter around page 100.



\subsection{Monte Carlo Localization}\label{sec:montecarlo}
\begin{figure}[htbp]
	\centering
		\includegraphics[page=70,trim={5cm 6cm 3cm 5.3cm},clip]{./Figures/fig.pdf}
		\rule{35em}{0.5pt}
	\caption[Simple example of localization with \gls{Particle} filter]{This is a simple example of the localization realized with a \gls{Particle} filter. It is similar to figure \ref{fig:loc_example}, but this time the belief is not a curve but instead represented by the distribution of \gls{Particle}s \citep[p.\ 251]{Thrun:2005:PR:1121596}.}
	\label{fig:particle_localization}
\end{figure}
The \gls{Particle} filter can be used for localization \citep[p.\ 252]{Thrun:2005:PR:1121596}. Figure \ref{fig:particle_localization} shows a simple example. This figure is similar to figure \ref{fig:loc_example}. But here the \gls{Particle}s and their weights are used to represent the belief distribution. In (a), (c) and (e) only the distribution of the \gls{Particle}s is shown, but not their weights. It shows how after some iterations the \gls{Particle}s are clustered around the positions that are most likely the real position. But this also involves the disadvantage that after some iterations the \gls{Particle}s will only be clustered around one spot, so we only observe this particular part of the whole belief. This is both an advantage and a disadvantage. It means we save computing power, because we only compute a small part of the whole belief distribution. But it also means that after some iterations we will only observe one particular point of the belief and disregard the rest. In certain situations this can lead to localization failures, which can't be resolved unless we tweak the algorithm. How this happens and how those situations can be salvaged will be explained in the following section \ref{sec:krp}.

There is \gls{ControlData} that is usually provided by an \gls{Odometer}. The measurement data is usually provided by sensors like \glspl{LaserRangeFinder}. In our specific case the Wi-Fi receiver will be another source of measurement data. 

A sample motion model and a measurement model are needed for the \Gls{MonteCarlo} to work. The sample motion model takes $u_t$ and $x_{t-1}^{[m]}$ as input. It takes the \gls{Odometer}'s data into account and samples new \gls{Particle}s based on that. In simpler terms this just means that the \gls{Particle}s are moved according to what we can infer from the \gls{ControlData} available. If a robot drove a few meters to the right the \gls{Particle}s should move to the right as well. This step does not take the measurement data into account yet.

That happens in the next step. The measurement model takes the newly sampled \gls{Particle} $x_t^{[m]}$ and measurement data $z_t$ and computes a weight. The higher the weight, the more likely is it according to the model that the \gls{Particle} is representative of the real pose. 

These models are different for different kinds of sensors. Later on we will show the used measurement model for the Wi-Fi receiver in-depth. 

\begin{algorithm}
\caption{Monte\_Carlo\_Localization \citep[p.\ 252]{Thrun:2005:PR:1121596}}
\label{alg:monte_carlo}
\begin{algorithmic}[1]
\Procedure{Monte\_Carlo\_Localization}{$X_{t-1},u_t,z_t,m$}
\State $\bar{X_t} = X_t = \emptyset$
\For{$m = 1$ to $M$}
\State $x_t^{[m]} = $ \textbf{sample\_motion\_model}$(u_t,x_{t-1}^{[m]})$
\State $w_t^{[m]} = $ \textbf{measurement\_model}$(z_t,x_{t-1}^{[m]},m)$
\State $\bar{X_t} = \bar{X_t} + \langle x_t^{[m]},w_t^{[m]}\rangle$
\EndFor
\For{$m = 1$ to $M$}
\State draw $i$ with probability $\propto w_t^{[i]}$
\State add $x_t^{[i]}$ to $X_t$
\EndFor
\State \Return $X_t$
\EndProcedure
\end{algorithmic}
\end{algorithm}

Algorithm \ref{alg:monte_carlo} is the \Gls{MonteCarlo}. Note how similar it is to the \gls{Particle} filter from algorithm \ref{particle_filter}. The only difference is how the new set of \gls{Particle}s is sampled in line 4 and how the weights are computed in line 5. This happens according to the sample\_motion\_model and the measurement\_model. 
\subsection{Global Localization}
\begin{figure}[htbp]
	\centering
		\includegraphics[width=\textwidth,height=\textheight,keepaspectratio]{./Figures/global_loc.png}
		\rule{35em}{0.5pt}
	\caption[Example of the \gls{GlobalLocalization}]{Example of \gls{GlobalLocalization} in our laboratory. The black circle represents the robot. The green arrows represent the position and orientation of the \gls{Particle}s. In the upper left pictures the \gls{Particle}s are spread over the entire state space. In the subsequent pictures the robot rotates to observe its environment. Slowly the \gls{Particle}s converge to the correct location on the map and thus it is possible to infer the correct position of the robot.}
	\label{fig:global_localization}
\end{figure}
There is local localization and there is \gls{GlobalLocalization}. In the local localization problem the position at the beginning is already known. The \gls{Particle}s are already clustered around the correct position at the beginning. Afterwards the only task is to track the position from there on out. In the \gls{GlobalLocalization} problem we have no information about the robot's location at the start.

A common approach to solve this problem is to spread the \gls{Particle}s over the entire map at the start. After some iterations of weighing and sampling the \gls{Particle}s they are supposed to be clustered around the real state. Sometimes this approach does not localize the robot in the right position. There are a number of circumstances that can increase the likelihood of such a global localization failure.

Many buildings have multiple rooms with similar structures. This can lead to the creation of multiple clusters and the pose being ambiguous. Many rooms are symmetric. This too can lead to clusters on multiple locations. 
\subsection{Kidnapped Robot Problem} \label{sec:krp}
The \gls{KidnappedRobotProblem} occurs when, while performing localization, the robot is taken and placed at a different location. It won't be able to register where it went in most cases. Wheeled robots usually use \gls{Odometer}s for the \gls{ControlData} and once the robot is taken up and not on the ground they won't register anything anymore. 

This leads to problems when the \gls{Particle}s are already clustered around one pose. When sampling new \gls{Particle}s this cluster will persist and there will not be any \gls{Particle}s on the rest of the map. To recover from such failures one can introduce a certain number of random \gls{Particle}s into the set \citep[p.\ 256]{Thrun:2005:PR:1121596}.

We could introduce a certain number of random \gls{Particle}s in every iteration. But we can also use the measurement model and the generated weights as an indicator how likely the current pose is. For this purpose we calculate the average weight of the current iteration \citep[p.\ 257]{Thrun:2005:PR:1121596}.
\begin{equation}
\dfrac{1}{M}\sum_{m=1}^{M}w_t^{[m]} \approx p(z_t|z_{1:t-1},u_{1:t},m)
\end{equation}
This already gives us a good idea, but we don't want to rely on only a single iteration to decide the induction of random \gls{Particle}s. There could be an unusually high amount of noise or other reasons for inaccurate measurements. So we take the average over multiple iterations into account. We keep track of two different values:
\begin{equation}\label{eq:decay}
\begin{aligned}
w_{slow} &= w_{slow} + \alpha_{slow}(w_{avg}-w_{slow})\\
w_{fast} &= w_{fast} + \alpha_{fast}(w_{avg}-w_{fast})
\end{aligned}
\end{equation}

$\alpha_{slow}$ and $\alpha_{fast}$ are decay rates and constants that have to be set beforehand. $w_{slow}$ is the long term measurement probability and $w_{fast}$ is the short term probability. This means that past weighting averages have a higher influence on $w_{slow}$ than they do on $w_{fast}$. 

During the resampling process random \gls{Particle}s are drawn with probability:
\begin{equation}\label{eq:quality}
\max\{0.0, 1.0-w_{fast}/w_{slow}\}
\end{equation}

The higher the long term probability $w_{slow}$ is compared to $w_{fast}$ the more likely it is that random \gls{Particle}s get introduced. That means the lower the recent weights are compared to the older ones, the higher the probability that random \gls{Particle}s are drawn \citep[p.\ 258-259]{Thrun:2005:PR:1121596}.
% TODO: Maybe add augmented mcl algorithm from page 258
\section{Wi-Fi}\label{sec:wifi}
A Wi-Fi signal is spread from a base station known as access point. Each signal carries a unique identifier called media access control (MAC) address, which can be used to set the different signals apart. The Service Set Identifier (\Gls{SSID}) on the other hand isn't unique and there are often multiple access points distributing Wi-Fi signals under the same \Gls{SSID}. This makes the \Gls{MAC-address} relevant and the \Gls{SSID} irrelevant for the Wi-Fi position estimation \citep{ieee802.11-2012}.

The Wi-Fi signal strength is measured in dBm. This is a logarithmic unit of measurement that is based on order of magnitude, rather than a linear scale. The reason for this is that the signal strengths deteriorate very quickly. 

According to our own experience the range of the signal is between about -90 dBm up to -40 dBm. The higher the number the better the signal. For our purposes the stability of the data transfer via the Wi-Fi connections is unimportant. We are only interested the received signal strength.

Wi-Fi signals are spread on different frequency ranges. The ranges usually used are 2.4 GHz and 5 GHz. These frequency bands are then further divided into channels. So each Wi-Fi signal is spread on a specific channel. The more Wi-Fi signals in a close range send on the same channel, the more they interfere with each other. 

Our goal is to fetch the MAC-address and the corresponding signal strength from every available Wi-Fi network, not only the one the robot is connected to. This can be achieved by actively scanning the different Wi-Fi channels. In order to receive the data a probe request is sent to the available access points. Each access point answers. This has to be done on each supported channel. Sending out the request, waiting for the answer and repeating that procedure for each channel means that it can take a few seconds to complete a whole scan. 

\section{Wi-Fi Sensor Model} \label{sec:gp}
\subsection{Overview}
As discussed earlier, in order for the Monte Carlo localization to work with a certain type of sensor, we need to provide a measurement model for it. Our goal is to create one for a Wi-Fi receiver. For this purpose we chose to create a map of the Wi-Fi signal strengths. Doing this for Wi-Fi signals is different from creating a map of obstacles for example. For an obstacle map a point on the map can only have 3 different states: free, occupied and unknown. But Wi-Fi signals change constantly from position to position. This makes it more complicated to create an accurate map. It isn't feasible to record the Wi-Fi signal strengths on every single point of the map. Like we discussed the scanning process alone will take a few seconds. Thus we choose to record a fair amount of data points and use \gls{Regression} to interpolate a likely value for the Wi-Fi signal strengths at every point on the map. Wi-Fi signals can be unpredictable in the way they spread. With no obstacles between robot and access point it is reasonable to expect the signal to constantly get weaker the further we go away. But in buildings there are walls and other obstacles that can obstruct and deflect the signal.

\Gls{GaussianProcess}es are a very flexible solution for this kind of problem. They are non-parametric and using \gls{Hyperparameter} optimization it is possible to fit them to the Wi-Fi signals. As we will show with a \Gls{GaussianProcess} it is possible to compute a Gaussian distribution for every coordinate on the map. So given a coordinate $x$ and $y$ we can compute a Gaussian distribution with a mean and variance. The resulting Gaussian distribution is a model of the probability of the occurrence of a signal strength at that coordinate. This property makes it very well suited to use it as measurement model. We are not interested in the most likely signal strength observed at a certain point on the map, but in the probability that a given signal strength is observed. This is needed in order to compute the weights for particles of the Monte Carlo localization, so a Gaussian process is very well suited for this task.

In the further subsections we will explain how \Gls{GaussianProcess}es work and how we can use them to create the Wi-Fi map and a measurement model.

\subsection{Gaussian Processes}\label{sec:gp_basics}
We have a dataset $D$ of $n$ observations of the form $D=\{(\mathbf{x_i},y_i)|i = 1,...,n\}$. In our example the $\mathbf{x_i}$ is a coordinate on the map and $y_i$ is the associated Wi-Fi signal strength. Now in \gls{Regression} we want to infer the function values for new inputs. So we want to get a function $f$ from the dataset $D$. 
But how do we infer function f? We need to make previous assumptions about the function because otherwise all functions that cross all training inputs and the corresponding values would be equally valid. The two common methods to achieve this are \citep[p.\ 2]{Rasmussen:2005:GPM:1162254}:
\begin{enumerate}
	\setlength\itemsep{0 em}
	\item Restrict function $f$ to certain classes of functions.
	\item Put \gls{Prior} probabilities on all possible functions.
\end{enumerate}
When using the first solution depending on what classes are chosen, they can be a bad fit. For example one can imagine that linear functions would be a bad fit for Wi-Fi data. They are not flexible enough. On the other hand it can happen that the classes chosen lead to overfitting when they are too flexible \citep[p.\ 2]{Rasmussen:2005:GPM:1162254}.

In the second solution we would need to put a \gls{Prior} on all possible functions. The possible functions are infinitely many, so this seems like a difficult task. This is where the \Gls{GaussianProcess} can help us. 

A stochastic process is a generalization of the probability distribution. While the probability distribution concerns scalars and vectors, the process concerns functions \citep[p.\ 2]{Rasmussen:2005:GPM:1162254}. Here we will focus on processes that are Gaussian. The reason for that is that is that it makes the computations required a lot easier.
 
One can think of a function $f(x)$ as a vector, where each entry specifies a value for a specific input $x$. These vectors would be infinitely large. When we want to infer a function value from a \Gls{GaussianProcess} at finitely many points it is as if we had taken the infinitely many other points taken into account, even though we ignore them \citep[p.\ 2]{Rasmussen:2005:GPM:1162254}. 
\begin{figure}[htbp]
	\centering
		\includegraphics[width=\textwidth,height=\textheight,keepaspectratio]{./Figures/gp_prior.png}
		\rule{35em}{0.5pt}
	\caption[\gls{Prior}]{An example of how the \gls{Prior} distribution of functions works. In (a) functions were drawn from the \gls{Prior}. As one can see they have similar characteristics. In (b) the \gls{Posterior} is shown. Two points were observed. The solid line is the mean prediction. The shaded region is twice the standard deviation at each input value $x$ \citep[p.\ 3]{Rasmussen:2005:GPM:1162254}}.
	\label{fig:gp_prior}
\end{figure}

To give a better intuition on how inferring the function $f$ from the dataset $D$ works we will give an example in form of figure \ref{fig:gp_prior}. As one can see in (a) the functions are drawn without any observations. In (b) two observations were made. The functions go through the observed values. The thick line signifies the mean prediction. But this method also gives us a variance at each input value. The closer the input value is to an observation, the lower the standard deviation indicated by the gray shadows. The obvious reason is that we are more certain about the true function value the closer we are to the observations we made.

When using \Gls{GaussianProcess}es the form of the drawn functions is determined by a covariance function \citep[p.\ 4]{Rasmussen:2005:GPM:1162254}. This function determines the functions that are considered for inference. So a different covariance function would have created a different mean and variance and the functions drawn from the \gls{Prior} in (a) would have looked different. This makes choosing a fitting covariance function important. A better fit produces a more accurate result. But covariance functions usually aren't static. There are parameters. But these can be learned, that is why they are called \gls{Hyperparameter}s. So while we still have to choose a fitting covariance function for the \Gls{GaussianProcess} they can be fit to the data by determining the right \gls{Hyperparameter}s. How to do this will be discussed later in this section. In the following section we will define what a \Gls{GaussianProcess} is and how it can be used for \gls{Regression}. 

%\subsubsection{Regression}
\paragraph{Regression}\mbox{}\\
\Gls{Regression} is a method to determine a relationship between a number of variables, using data that was recorded beforehand. So in our example we want to know the relationship between the Wi-Fi signal strength and the coordinates on the map. 
We are going to use Gaussian process regression to predict the likelihood of Wi-Fi signals at certain coordinates based on the Wi-Fi signals that we observed beforehand. To explain how we do this with \Gls{GaussianProcess}es we are going to start with a simpler \gls{Regression} and go on from there. 

We are going to look at the linear model from a Bayesian perspective. 

Once again we have a \gls{TrainingSet} $D$ of $n$ observations in the form of $D = \{(\mathbf{x_i},y_i)|i=1,...,n\}$. To make things easier we are going to introduce vector $\mathcal{X}$ and $y$ to form $D = (\mathcal{X},y)$.

Now the standard linear \gls{Regression} model with Gaussian noise has the following form:
\begin{equation}\label{linearmodel}
\begin{aligned}
f(\mathbf{x}) &= \mathbf{x^T}\mathbf{w}\\
y &= f(\mathbf{x}) + \varepsilon
\end{aligned}
\end{equation}
Here $f(\mathbf{x})$ is the function value. $\mathbf{w}$ is a vector of weights, which act as the parameters of the linear function. $y$ are the observed function values. Because the real values won't be exactly on the graph of $f(\mathbf{x})$ the noise term $\varepsilon$ is added \citep[p.\ 8]{Rasmussen:2005:GPM:1162254}. 

The noise term is of the form of a Gaussian distribution with zero mean and variance $\sigma_n^2$.
\begin{equation}\label{noise_term}
\varepsilon \sim \mathcal{N}(0,\sigma_n^2)
\end{equation}
From this we infer the likelihood $p(\mathbf{y}|\mathbf{X},\mathbf{w})$, so the probability density of the observations $\mathbf{y}$ given the weights $\mathbf{w}$ and the training inputs $\mathbf{X}$ \citep[p.\ 9]{Rasmussen:2005:GPM:1162254}.
\begin{equation}\label{eq:likelihood}
\begin{aligned}
p(\mathbf{y}|\mathbf{X},\mathbf{w}) &= \prod_{i=1}^{n}p(y_i|\mathbf{x_i},\mathbf{w}) = \prod_{i=1}^{n}\dfrac{1}{\sqrt{2\pi\sigma_n}}\exp\big({-\dfrac{(y_i-\mathbf{x_i}^T\mathbf{w})^2}{2\sigma_n^2}}\big)\\
& = \dfrac{1}{(2\pi\sigma_n)^{n/2}}\exp\big({-\dfrac{1}{2\sigma_n^2}|\mathbf{y}-\mathbf{X}^T\mathbf{w}|^2}\big) = \mathcal{N}(\mathbf{X}^T\mathbf{w},\sigma_n^2I)
\end{aligned}
\end{equation}
Here $I$ is the identity matrix. We can form the likelihood to a Gaussian distribution with mean $\mathbf{X}^T\mathbf{w}$, which is $f(x)$, and a variance of $\sigma_n^2I$, which resembles the noise term $\varepsilon$. 

Because we use the Bayesian formalism we need to specify a \gls{Prior} over the weights. This \gls{Prior} expresses what we believe their nature is like before we look at the observations. For this we use another Gaussian distribution. It will have a mean of zero and the variance is the covariance matrix $\Sigma_p$ \citep[p.\ 9]{Rasmussen:2005:GPM:1162254}.
\begin{equation}\label{eq:weight_prior}
\mathbf{w} \sim \mathcal{N}(0,\Sigma_p)
\end{equation}

To be able to predict new values, we need to know the weights. This means we want to infer the value of the weights from the \gls{TrainingSet}. This is called the \gls{Posterior} and has the form $p(\mathbf{w}|\mathbf{X},\mathbf{y})$. We can use Bayes' theorem in order to achieve this.
\begin{equation}\label{eq:bayestheorem}
\begin{aligned}
\text{posterior} &= \dfrac{\text{likelihood} \cdot \text{prior}}{\text{marginal likelihood}}\\
p(\mathbf{w}|\mathbf{y},\mathbf{X}) &= \dfrac{p(\mathbf{y}|\mathbf{X},\mathbf{w})p(\mathbf{w})}{p(\mathbf{y}|\mathbf{X})}
\end{aligned}
\end{equation}
$p(\mathbf{y}|\mathbf{X})$ is independent of the weights and therefore it is just a normalizing constant. Using the theorem of total probability it takes on the following form:
\begin{equation}\label{eq:nc_tp}
p(\mathbf{y}|\mathbf{X}) = \int p(\mathbf{y}|\mathbf{x},\mathbf{w})p(\mathbf{w})d\mathbf{w}
\end{equation}
Now leaving out the normalizing constant and only concentrating on the likelihood and \gls{Prior} we are able to form the \gls{Posterior} into a normal distribution, by using equations \ref{eq:likelihood} and \ref{eq:weight_prior}.
\begin{equation}\label{eq:posterior_distri}
\begin{aligned}
p(\mathbf{w}|\mathbf{X},\mathbf{y}) &\propto \exp\big(-\dfrac{1}{2\sigma_n^2}(\mathbf{y}-\mathbf{X}^T\mathbf{w})^T(\mathbf{y}-\mathbf{X}^T\mathbf{w})\big)\exp \big(-\dfrac{1}{2}\mathbf{w}^T\Sigma_p^{-1}\mathbf{w}\big)\\
&\propto \exp \big(-\dfrac{1}{2}(\mathbf{w}-\bar{\mathbf{w}})^T(\dfrac{1}{\sigma_n^2}\mathbf{X}\mathbf{X}^T+\Sigma_p^{-1})(\mathbf{w}-\bar{\mathbf{w}})\big)
\end{aligned}
\end{equation}
Here $\bar{\mathbf{w}} = \sigma_n^{-2}(\sigma_n^{-2}\mathbf{X}\mathbf{X}^T+\Sigma_p^{-1})^{-1}\mathbf{X}\mathbf{y}$. Now the resulting Gaussian distribution is the following:
\begin{equation}\label{eq:posterior_gauss}
p(\mathbf{w}|\mathbf{X},\mathbf{y}) \sim \mathcal{N}(\bar{\mathbf{w}}=\dfrac{1}{\sigma_n^2}A^{-1}\mathbf{X}\mathbf{y},A^{-1})
\end{equation}
with $A = \sigma_n^{-2}\mathbf{X}\mathbf{X}^T + \Sigma_p^{-1}$ \citep[p.\ 9]{Rasmussen:2005:GPM:1162254}.

Now we want to predict new values. We have a new input $\mathbf{x_*}$ with the function value $f_* \triangleq f(\mathbf{x_*})$. In order to achieve this we use the \gls{Posterior} distribution and average over all possible parameter values \citep[p.\ 11]{Rasmussen:2005:GPM:1162254}.
\begin{equation}\label{eq:predictive_distribution}
\begin{aligned}
p(f_*|\mathbf{x_*},\mathbf{X},\mathbf{y}) &= \int p(f_*|\mathbf{x_*},\mathbf{w})p(\mathbf{w}|\mathbf{X},\mathbf{y})d\mathbf{w}\\
&= \mathcal{N}(\dfrac{1}{\sigma_n^2}\mathbf{x_*}^TA^{-1}\mathbf{X}\mathbf{y},\mathbf{x_*}^TA^{-1}\mathbf{x_*})
\end{aligned}
\end{equation}
This is called the predictive distribution. With this distribution one can predict the function values for new inputs $\mathbf{x_*}$. But of course this solution is still very limited. It still only considers linear functions and won't be flexible enough for the Wi-Fi data. 

But there is a trick we can apply in order to fix this flaw. We can simply project the inputs into high dimensional space by using a set of basis functions and apply the linear \gls{Regression} model in that space. A simple example would be $\phi(x) = (1,x,x^2,x^3,...)$. As long as the function is independent of $\mathbf{w}$ one can still apply the linear \gls{Regression} to it \citep[p.\ 11]{Rasmussen:2005:GPM:1162254}.

We will look closer at the basis function at a later point, for now it is given by $\phi(x)$. This changes the model for linear \gls{Regression} as follows \citep[p.\ 12]{Rasmussen:2005:GPM:1162254}.
\begin{equation}\label{eq:basis_function}
f(\mathbf{x}) = \phi(\mathbf{x})^T\mathbf{w}
\end{equation}
Now applying the basis function to predictive distribution from equation \ref{eq:predictive_distribution} can easily be done, by just applying the basis function to the inputs. 
\begin{equation}\label{eq:prediction_distri_basis_function}
f_*|\mathbf{x_*},\mathbf{X},\mathbf{y} \sim \mathcal{N}(\dfrac{1}{\sigma_n^2}\phi(\mathbf{x_*})^TA^{-1}\Phi\mathbf{y}, \phi(\mathbf{x_*})^TA^{-1}\phi(\mathbf{x_*}))
\end{equation}
Here $\Phi = \Phi(\mathbf{X})$ and $A = \sigma_n^{-2}\Phi\Phi^T+\Sigma_p^{-1}$.

One drawback here is $A^{-1}$. We would have to invert $A$ which is a $N\times N$ matrix, where $N$ is the size of the feature space. But we can rewrite the formula \citep[p.\ 12]{Rasmussen:2005:GPM:1162254}. 
\begin{equation}\label{eq:final_prediction_distri}
f_*|\mathbf{x_*},\mathbf{X}, \mathbf{y} \sim \mathcal{N}(\phi_*^T\Sigma_p\Phi(K+\sigma_n^2I)^{-1}\mathbf{y}, \phi_*\Sigma_p\phi_*-\phi_*^T\Sigma_p\Phi(K+\sigma_n^2I)^{-1}\Phi^T\Sigma_p\phi_*)
\end{equation}
Here $\phi(\mathbf{x_*}) = \phi_*$ and $K = \Phi^T\Sigma_p\Phi$.

%ToDo: Add link to cov function, or kernel function. Explain \gls{Hyperparameter}s and their role.
Now we will introduce function $k(\mathbf{x_p}, \mathbf{x_q})$ which is called the covariance or \gls{Kernel} function. 
\begin{equation}\label{kernel}
\begin{aligned}
k(\mathbf{x_p}, \mathbf{x_q}) &= \phi(\mathbf{x_p})^T\Sigma_p\phi(\mathbf{x_q})\\
&= \psi(\mathbf{x_p}) \cdot \psi(\mathbf{x_q})
\end{aligned}
\end{equation}
The fact that we can rewrite this function as a dot product makes it possible to apply the so called \gls{Kernel} trick \citep[p.\ 12]{Rasmussen:2005:GPM:1162254}. We can define a function $k(\mathbf{x_p},\mathbf{x_q})$ and then replace all occurrences by it. The \gls{Kernel} trick is used to lift the inputs into a higher dimensional space, while still making all computations in the input space. This saves a lot of computation time and memory. So we are going to place the \gls{Kernel} function where we can in equation \ref{eq:final_prediction_distri}.
\begin{equation}\label{prediction_distri_with_kernel}
f_*|\mathbf{x_*},\mathbf{X}, \mathbf{y} \sim \mathcal{N}(\mathbf{k_*}^T(K+\sigma_n^2I)^{-1}\mathbf{y}, k(\mathbf{x_*},\mathbf{x_*})-\mathbf{k_*}^T(K+\sigma_n^2I)^{-1}\mathbf{k_*})
\end{equation}
With $\mathbf{k_*}$ denoting a vector of covariances between the test point and the training points, and $K$ being the covariance matrix of all training points, so $K = k(\mathbf{X}, \mathbf{X})$.

Equation \ref{prediction_distri_with_kernel} now enables us, given the training data $\mathbf{X}$ and $\mathbf{y}$, to compute a Gaussian distribution for a new data point $\mathbf{x_*}$. If $\mathbf{X}$ are previously recorded coordinates and $\mathbf{y}$ the corresponding signal strengths, then given some coordinate $\mathbf{x_*}$, we would be able to compute a Gaussian distribution. This distribution can be used to infer how likely it is that a signal strength was observed at that coordinate $\mathbf{x_*}$. This is exactly what we need to compute the weights for the \Gls{MonteCarlo} from section \ref{sec:montecarlo}.

Now we have everything we need to fully define the \Gls{GaussianProcess} and then predict values from it. 

According to \citet[p.\ 13]{Rasmussen:2005:GPM:1162254} a \Gls{GaussianProcess} is fully defined as follows:
\begin{equation}\label{eq:GP}
\mathcal{GP}(m(\mathbf{x}),k(\mathbf{x},\mathbf{x'}))
\end{equation}
The mean function $m(\mathbf{x})$ is usually defined as 0. 

The \gls{Kernel} function $k(\mathbf{x},\mathbf{x'})$, also called covariance function, is an important aspect of the \Gls{GaussianProcess}. There are many possibilities to choose from for \gls{Kernel} functions. The functions considered as \gls{Kernel} functions have to be symmetric and positive semi-definite. 

But while there are many functions that could be used as \gls{Kernel}, we are going to focus on one of the most popular ones. The radial basis function \gls{Kernel}. (RBF \gls{Kernel}). \citet{ferris2006gaussian} showed that this kernel is very well suited for the usage with Wi-Fi signal strengths. 
\begin{equation}\label{eq:rbf}
k(x_p,x_q) = \sigma_f^2\exp(-\dfrac{1}{2l^2}(x_p-x_q)^2)+\sigma_n^2\delta_{pq}
\end{equation} 
Here $\delta_{pq}$ is the Kronecker delta, that is 1 whenever $p=q$ and 0 else. This particular \gls{Kernel} function results in a very smooth graph. There are three so called \gls{Hyperparameter}s. These are variables, but they don't have to be set manually, but can actually be learned. How will be discussed in the next section. 
The hyperparameters here are the lengthscale $l$, the signal variance $\sigma_f^2$ and the noise variance $\sigma_n^2$. 

%\subsubsection{Hyperparameter Optimization}
\paragraph{Hyperparameter Optimization}\mbox{}\\
The \Gls{GaussianProcess} \gls{Regression} is a parameterless \gls{Regression} method. But we have to deal with so called \gls{Hyperparameter}s. The advantage here is that we can use optimization algorithms in order to find values that are working well. In order to find those values we take the \gls{LogLikelihood} and maximize it. The \gls{LogLikelihood} function gives us an indication how well the \Gls{GaussianProcess} fits the data. The higher the value of the \gls{LogLikelihood} function the closer the \Gls{GaussianProcess} resembles the given data.

The \gls{LogLikelihood} is defined by the following function \citep[p.\ 113]{Rasmussen:2005:GPM:1162254}.
\begin{equation} \label{eq:ll}
log\,p(\mathbf{y}|\mathbf{X},\theta) = -\dfrac{1}{2}\mathbf{y}^T(K+\sigma^2_nI)^{-1}\mathbf{y}-\dfrac{1}{2}log\,|K+\sigma^2_nI|-\dfrac{n}{2}log\,2\pi
\end{equation}

Here $\theta$ are the \gls{Hyperparameter}s $l$, $\sigma_n^2$ and $\sigma_f^2$. In order to use optimization algorithms we also need the partial derivatives of the function \citep[p.\ 114]{Rasmussen:2005:GPM:1162254}.
\begin{equation}\label{eq:lld}
\dfrac{\partial}{\partial\theta_j}log\,p(\mathbf{y}|\mathbf{X},\theta) = \dfrac{1}{2}tr\bigg((K^{-1}\mathbf{y}) (K^{-1}\mathbf{y})^T \dfrac{\partial K}{\partial \theta_j}\bigg)
\end{equation}

So, we need a method that uses the information from the \gls{LogLikelihood} function and its gradient, and maximizes the value by adjusting the \gls{Hyperparameter} values. 
Some examples of algorithms that are suitable to optimize the \gls{Hyperparameter}s are the \gls{GradientDescent} \citep{Shewchuk:1994:ICG:865018}, \Gls{BFGS} or \Gls{L-BFGS} \citep{liu1989limited}. \citet{blum2013optimization} propose to use resilient backpropagation (\Gls{Rprop}) to solve this problem. They show that the algorithm has a similar performance as \Gls{L-BFGS}, but it has the advantage over \Gls{L-BFGS} and similar methods that it is easier to implement.

Like most methods the algorithm requires not only the function itself, but also the gradient. However it doesn't use the second order derivatives or an approximation thereof, which leads to shorter computation times per iteration \citep{blum2013optimization}. 

In every iteration the \gls{Hyperparameter}s $\theta$ are updated depending on the sign of the derivative:
\begin{equation}
\theta_i^{(t+1)} = \theta_i^{(t)} - sign\bigg(\dfrac{\partial J^{(t)}}{\partial \theta_i}\bigg) \Delta_i^{(t)}
\end{equation}
$\Delta_i$ is the update-value. Depending on the change of the sign the \gls{Hyperparameter}, $\Delta_i$ is either increased by a factor of $\eta^+ > 1$ or decreased by a factor $0 < \eta^- < 1$. 
\begin{equation}
\Delta_i^{(t)} = 
\begin{cases}
\begin{aligned}
\eta^+\cdot\Delta_i^{(t-1)} &\text{, if } \dfrac{\partial J}{\partial \theta_i}^{(t-1)}\cdot\dfrac{\partial J}{\partial \theta_i}^{(t)} > 0 \\
\eta^-\cdot\Delta_i^{(t-1)} &\text{, if } \dfrac{\partial J}{\partial \theta_i}^{(t-1)}\cdot\dfrac{\partial J}{\partial \theta_i}^{(t)} < 0\\
\Delta_i^{(t-1)} &\text{, else}
\end{aligned}
\end{cases}
\end{equation}

The initial update value is set to $\Delta_0$ and is bounded by $\Delta_{min}$ and $\Delta_{max}$. The parameters have to be specified, but there are values that work for most cases \citep{blum2013optimization}. 

So in order to get good values for the \gls{Hyperparameter}s we need to apply the algorithm to the partial derivatives as specified in equation \ref{eq:lld}. Like most optimization algorithms it tries to minimize the function value, but we need to maximize it. The higher the \gls{LogLikelihood} is the better the fit. So we simply use the negative \gls{LogLikelihood} and its negative partial derivatives. 

The algorithm will run for a set number of iterations or until every partial derivative is 0. At every iteration we check the negative \gls{LogLikelihood} and compare it with the best result we found yet. If the new result is smaller we save the current \gls{Hyperparameter}s. At the end of this process the \Gls{GaussianProcess} will fit well to the training data \citep{blum2013optimization}.

At the beginning of this chapter we explained the basics of localization. We moved from the uncertainty involved in robotics to the \gls{RecursiveStateEstimation} to its implementation with a \gls{Particle} filter. At the end of section \ref{sec:localization} we showed the \Gls{MonteCarlo} that uses \gls{Particle} filters. For the \Gls{MonteCarlo} to actually work we need sensor models. For a Wi-Fi receiver \Gls{GaussianProcess}es are well suited. 
So, in the second half of this chapter we determined what a \Gls{GaussianProcess} is and showed how it can be used for \gls{Regression} by deriving it from the linear \gls{Regression}. Even though the Gaussian process regression is a parameterless method, there are hyperparameters that have to be optimized. We explained how this can be done using \Gls{Rprop}. 

In the next chapter we will use this knowledge to explain our implementation of the Wi-Fi position estimation, by using \Gls{GaussianProcess}es and the \Gls{MonteCarlo}. 

\chapter{Implementation} % Write in your own chapter title
\label{Chapter3}
%\lhead{Chapter 3. \emph{Implementation}} % Write in your own chapter title to set the page header
\fancyhead[LE,RO]{Chapter 3. \emph{Implementation}}
This chapter discusses the implementation of the parts needed for a functioning Wi-Fi position estimation. In section \ref{sec:ros} we will introduce the Robot Operating System(ROS) and explain its basic components.
In section \ref{sec:publisher} we will explain how the required Wi-Fi data gets fetched and published for other parts of the program.
Section \ref{sec:amcl} gives a short overview about the used implementation of the \Gls{MonteCarlo} for the 2D \gls{LaserRangeFinder}.
In section \ref{sec:data_coll} we will show how the published Wi-Fi and position data from the robot is collected and stored for further use with the Wi-Fi position estimation.
The section \ref{sec:gausspr} details the implementation of the \Gls{GaussianProcess} for the Wi-Fi position estimation.
And at last section \ref{sec:wifiposest} will tie the previous chapters together and explain the Wi-Fi position estimation as a whole.

\section{ROS}\label{sec:ros}
\begin{figure}[htbp]
	\centering
		\includegraphics[width=\textwidth,height=\textheight,keepaspectratio]{./Figures/dia/ros.eps}
		\rule{35em}{0.5pt}
	\caption[Diagram of basic concepts in ROS]{Relationship of the basic parts of the ROS environment.}
	\label{fig:ros_architecture}
\end{figure}
The Wi-Fi position estimation was implemented with the ROS-Framework and run on a TurtleBot 2 (http://www.turtlebot.com/, accessed on 02.11.2016). ROS \citep{Fernandez:2015:LRR:2876174} \citep{288} is short for Robot Operating System. It supports a wide variety of robots and was designed so that different robots and environments have a common basis to make it easier to share. 

The software in ROS is divided into packages. So each package usually has a different functionality or purpose. An example used in this project is the adaptive Monte Carlo localization (AMCL) package and the created package for the Wi-Fi position estimation. These packages contain nodes. The nodes are processes. Before the nodes can be executed a master has to be started. On this master the nodes are registered, so that different nodes can see each other and can communicate with each other. 

Another important part of the ROS-framework for basic functionality is the parameter server. It can be used in order to supply a node with manually set parameters when the node is started. This makes it easier to customize the nodes. 

For the purpose of communication between the nodes there are topics and services. Topics can either be published or subscribed to. For example, there is the AMCL node, that publishes a pose estimation and a Wi-Fi publisher node, that publishes the Wi-Fi signal strengths. The Wi-Fi data recorder is subscribed to these nodes so it can save the signal strengths and the corresponding position on the map. A service can be offered or called by a node. For example AMCL offers \gls{GlobalLocalization} via service. Services are only performed, whenever they are called, while the data published on topics usually gets published constantly.

The ROS framework offers a wide variety of packages for a wide variety of robots. The advantages of this are that we don't have to build everything from the ground up. There is already existing software that is maintained by a large community. Different packages can be re-used for different robots, as long as these robots run ROS. 

There are some other ROS packages and nodes that were used in this project. The rosbag package (http://wiki.ros.org/rosbag, accessed on 31.10.2016) can be used to record data and store it to play it back later. For the nodes there is no difference if the robot is actually producing them right now or if they are just played back from a rosbag. One of its uses was in carrying out experiments to apply different algorithms to the same set of data. 

Another package we used is RViz (http://wiki.ros.org/rviz, accessed on 31.10.2016). It was used to visualize the Wi-Fi data and the Gaussian process for various figures in this thesis. 

Starting these nodes can be done via the command line. The typical way to do this is the command ``rosrun package\_name node\_name''. Another option is creating a launch file and starting it with the roslaunch-package (http://wiki.ros.org/roslaunch, accessed on 01.11.2016). These files are written in the XML-format. They can contain parameters that are used by the node launched. A launch file can be started using the terminal command ``roslaunch package\_name launch\_file.launch''.

\section{Wi-Fi Data Publisher}\label{sec:publisher}
The first step to realize the position estimation is to actually get the Wi-Fi data. For this purpose we use an active Wi-Fi scan. We wait for the result and handle the data. Then the \gls{SSID}s, \Gls{MAC-address}es and signal strengths are published. This procedure gets repeated until the node is stopped. As already discussed the scan can take up to a few seconds. Of course this means that the rate at which the node publishes new Wi-Fi data is limited by the time the scan takes as well.

The node publishes a topic called ``/wifi\_state''. It contains the Wi-Fi signal strength in dBm, the MAC-addresses and the SSIDs. 

In order to get this data the following terminal command is used:
\begin{lstlisting}[language=bash, basicstyle=\small]
  $ sudo iw dev wlan0 scan 
\end{lstlisting}
Here wlan0 is the used network interface. This value can be different on other systems. This command starts a scan to find all Wi-Fi networks that can be found. It provides a lot of information about every network and among them is the signal strength, the MAC-address and the frequency.

Without further specification the command will scan all channels, but it is also possible to specify a frequency that corresponds to a Wi-Fi channel. This will limit the Wi-Fi networks that can be found, but it is also a lot faster. Here is a simple example:
\begin{lstlisting}[language=bash, basicstyle=\small]
  $ sudo iw dev wlan0 scan freq 2412
\end{lstlisting}
Different frequencies correspond to different Wi-Fi channels. 2412 MHz for example is the frequency of channel 1. 

In our experience the complete scan takes about 3-5 seconds, depending on the Wi-Fi receiver used. The scan using a specified frequency only takes a fracture of a second. A short test resulted in approximately 20 scans per second, but once again this depends on the Wi-Fi receiver used. Since we want the data from all available Wi-Fi access points only the complete scan was used.

\section{AMCL}\label{sec:amcl}
We use the Adaptive Monte Carlo Localization (AMCL) package \sloppy{(http://wiki.ros.org/amcl, accessed on 25.10.2016)} for the localization with the 2D range finder. The package contains an implementation of the Monte Carlo localization and furthermore provides the needed measurement model for the laser range finder. The package implements the following specific algorithms from \citet{Thrun:2005:PR:1121596}:
\begin{itemize}
\item sample\_motion\_model\_odometry: A sample motion model using the odometry.
\item beam\_range\_finder\_model: A measurement model using the laser range finder or Kinect.
\item likelihood\_field\_range\_finder\_model: A measurement model using the laser range finder or Kinect.
\item Augmented\_MCL: The Monte Carlo Localization with the method from section \ref{sec:krp}
\item KLD\_Sampling\_MCL: The Monte Carlo Localization with a mechanism that varies the number of particles depending on how certain the Monte Carlo Localization is about the localization.
\end{itemize}
The AMCL node needs a map of the environment the robot is supposed to be localized in. The map is created beforehand by using a package like gmapping (http://wiki.ros.org/gmapping, accessed on 01.11.2016) for example. The map can be provided to AMCL by using the map\_server package \\(http://wiki.ros.org/map\_server, accessed on 01.11.2016). So a map\_server node loading the correct map has to be started before starting the AMCL node.
When started, AMCL publishes the pose of the robot determined by the localization. This pose contains the $x$ and $y$ coordinate and the orientation as a quaternion on the given map. The poses also contain a covariance matrix, that represents the spread of the particles around the determined position. 

It should be noted that no matter how the robot is moved, AMCL will continue to localize the robot as long as the wheels are used to move it. So only when the robot is picked up and moved, AMCL is not able to track the position.

The package has multiple use cases in this project. For one it is used for data collection, so that we know where on the map the data was recorded. It is also used in conjunction with the Wi-Fi position estimation, to test how well it works as an aid for the \gls{GlobalLocalization}. 

The last use case is to use the Wi-Fi position estimation in case of a localization failure. We can use the method discussed in section \ref{sec:krp} as an indicator if there is a localization failure. Using the value from equation \ref{eq:quality} we can infer the quality of the localization. Usually AMCL does compute this value, but does not publish it so that other nodes can use it. AMCL was slightly modified, so that it publishes that value under the identifier ``/amcl\_failure\_probability''. Here the higher the given value, the more likely it is that there is a localization failure.

\section{Data Collection}\label{sec:data_coll}
\begin{figure}[htbp]
	\centering
		\includegraphics[width=\textwidth,height=\textheight,keepaspectratio]{./Figures/data_collector_2.eps}
		\rule{35em}{0.5pt}
	\caption[Diagram of wifi\_data\_collector]{Overview over the topics the Wi-Fi\_data\_collector subscribed to and the related nodes. The /amcl\_pose and /wifi\_state topics are publishing the data that is supposed to be stored. The /amcl\_failure\_probability topic gives us an indication if there is a localization failure and in that case when a certain threshold is exceeded it stops recording the data. The data is then stored in a number of CSV files.}
	\label{fig:data_collector}
\end{figure}
In order to collect the data, the aforementioned Wi-Fi data publisher is used in combination with AMCL. We need to know the position of the robot on the map, if we want to use the data later to build a map of the signal strengths. AMCL publishes the estimated pose of the robot and the Wi-Fi data publisher publishes the \Gls{MAC-address}es and the measured signal strengths. Whenever new Wi-Fi data is published the data collection node stores that data together with the last published pose in comma separated value (CSV) files. For each \Gls{MAC-address} there is a separate file. 

There are some options to customize the data collector. One can choose if the robot is supposed to collect the data constantly or only in case it stands still. Another option is to load existing CSV files, so that the data collector can add new data points to them. 

In order to collect the data, the mobile robot needs to move around. In order to do this the move\_base package (http://wiki.ros.org/move\_base, accessed on 03.11.2016) can be used. One can provide a goal consisting of coordinates from the used map and an orientation, then the robot will move to this location. Another way to move the robot is the ``turtlebot\_teleop'' package \sloppy{(http://wiki.ros.org/turtlebot\_teleop, accessed on 03.11.2016)}. Here one has to control the movement of the robot with peripherals like keyboards or joysticks. 
\begin{figure}[htbp]
	\centering
		\includegraphics[width=\textwidth,height=\textheight,keepaspectratio]{./Figures/wifi_data.png}\\%./Figures/gp_wifi_comparison_2.png}
		\rule{35em}{0.5pt}
	\caption[Wi-Fi data]{Both pictures represent Wi-Fi data from one access point, recorded on the hallway of the TAMS research group at the University of Hamburg. Above each dot is a certain recording at the position on the map. Beneath we interpolated the data, to highlight the behavior of the signal strength. The colors from strong to weak signal strength: yellow, light blue, red, purple, blue.}
	\label{fig:wifi_data}
\end{figure}
 
\section{Gaussian Process}\label{sec:gausspr}

\begin{figure}[htbp]
	\centering
		\includegraphics[width=\textwidth,height=\textheight,keepaspectratio]{./Figures/gp_mean_variance.png}\\%./Figures/gp_wifi_comparison_2.png}
		\rule{35em}{0.5pt}
	\caption[Wi-Fi data and the corresponding \Gls{GaussianProcess}]{The picture above represents the mean of a \Gls{GaussianProcess} that was trained for the data from figure \ref{fig:wifi_data}. The colors from high to low: red, yellow, green, light blue, dark blue, purple. Below is the corresponding variance of the same process. As one can see the variance is very low close to the region where the data was recorded. This happens because we are more confident about the signal strengths in these regions. The further away we go, the higher the variance.}
	\label{fig:gp_mean_var}
\end{figure}
The model from section \ref{sec:gp} was used to implement the \Gls{GaussianProcess}. A change was made to the formula for the \gls{Kernel}. The reason for this is that we want to prevent the \gls{Hyperparameter}s from taking on negative values. This means we either have to apply an optimization algorithm that works for constrained problems, or make sure the parameters can't take on negative values. The first solution would mean we would have to use a more complicated algorithm, therefore we decided to take the second approach. 

The covariance function from equation \ref{eq:rbf} is changed to the following form:
\begin{equation}
cov(y_p,y_q) = \sigma_f \exp\bigg(-\dfrac{|x_p-x_q|^2}{l}\bigg) + \sigma_n \delta_{pq}
\end{equation}
In order to make sure that the \gls{Hyperparameter}s can only take on positive values they are substituted by the following terms:
\begin{equation}
\begin{aligned}
\sigma_f &= exp(2\theta_1)\\
l &= exp(\theta_2)\\
\sigma_n &= exp(\theta_3)
\end{aligned}
\end{equation}

This leads to changes for the gradient too:
\begin{equation}
\begin{aligned}
\dfrac{\partial cov}{\partial \theta_1} &= 2\sigma_f exp\bigg(-\dfrac{|x_p-x_q|^2}{l}\bigg)\\
\dfrac{\partial cov}{\partial \theta_2} &= -\sigma_f exp\bigg(-\dfrac{|x_p-x_q|^2}{l}\bigg)\bigg(-\dfrac{|x_p-x_q|^2}{l}\bigg)\\
\dfrac{\partial cov}{\partial \theta_3} &= \sigma_n\delta_{pq}
\end{aligned}
\end{equation}

Now the optimization algorithm is applied to $\theta_1$, $\theta_2$ and $\theta_3$. When these become negative, $\sigma_f$, $l$ and $\sigma_n$ will stay positive. 

The \Gls{GaussianProcess} is not a node, but a simple class. In order to create a \Gls{GaussianProcess} usually a path to a folder with CSV-files is provided. These files contain training data, that is, the positions and corresponding signal strengths. Each \Gls{MAC-address} usually has its own CSV-file named after itself, so that the \Gls{GaussianProcess} can distinguish them.

The implementation of the \Gls{GaussianProcess} was done with the Eigen-library (http://eigen.tuxfamily.org/, accessed on 28.10.2016). There are many matrix- and vector-operations needed and the Eigen-library is well suited for these kind of problems, as it provides matrix- and vector-classes with a number of useful algorithms for them. 

For the optimization of the \gls{Hyperparameter}s the Rprop-algorithm detailed in section \ref{sec:gp_basics} was implemented in a separate class. So once a \Gls{GaussianProcess} was created, the Rprop-class can be used in order to train the \Gls{GaussianProcess}, so that it fits the training data. 

Another aspect to consider is the nature of the Wi-Fi signal strength data and how it affects the Gaussian process. We are using a constant mean function of 0. In practice that means that when moving away from any training data, the mean of the Gaussian process will converge to 0. 

We decided to normalize the data to a range from 0 to 1. In order to do this we decided to use a maximum of 0 dBm and a minimum of -100 dBm. In our experience these values are not reached in real world scenarios. This is especially important for the minimum. A value of -100 dBm would be 0 after normalization and this would mean it would be indistinguishable from a value far outside the range of the training data. This is a scenario we want to avoid. Another reason for the normalization is that the used optimization algorithm, the already discussed Rprop-algorithm, works better with the normalized values. 
The resulting operation for the normalization is the following:
\begin{equation}
\begin{aligned}
\mathbf{y'} &= \dfrac{\mathbf{y}-(-100)}{0 - (-100)}\\
&= \dfrac{\mathbf{y} + 100}{100}
\end{aligned}
\end{equation}
This operation is both applied to the training data and to the new incoming data, when a weight for a particle is computed. 

\section{Wi-Fi Position Estimation}\label{sec:wifiposest}
\begin{figure}[htbp]
	\centering
		\includegraphics[width=\textwidth,height=\textheight,keepaspectratio]{./Figures/wifi_pos_est.eps}
		\rule{35em}{0.5pt}
	\caption[Diagram of the Wi-Fi\_localization]{A basic diagram of the Wi-Fi\_localization and the subscribed and published topics.}
	\label{fig:ros_localization}
\end{figure}
\begin{figure}[htbp]
	\centering
		\includegraphics[width=\textwidth,height=\textheight,keepaspectratio]{./Figures/wifi_pos_est.png}
		\rule{35em}{0.5pt}
	\caption[Wi-Fi position estimation example]{Example of using the Wi-Fi position estimation for \gls{GlobalLocalization} in our laboratory. The black circle represents the robot and the green arrows the position and orientation of the particles of AMCL using the 2D laser range finder. In the upper left picture the Wi-Fi position estimation was performed already. Compared to figure \ref{fig:global_localization} one can see that the particles are a lot more concentrated and not spread over the entire state space. The position is off by a certain margin. In the following pictures the robot is turning around itself to observe its environment with the 2D \gls{LaserRangeFinder}. The particles are quickly clustered around the correct location and thus the correct position of the robot can be inferred.}
	\label{fig:wifi_pos_est_example}
\end{figure}
The Wi-Fi position estimation uses the data that was collected to create a number of \Gls{GaussianProcess}es. For every network a new Gaussian process is created.

Since we only try to estimate the position we basically only replicate the first step of the global \Gls{MonteCarlo}. So particles are spread across the map. 

The Wi-Fi data publisher node will provide us with the current Wi-Fi signal strengths of the different networks. We will use this data with the coordinates from the particles to compute the weights for each particle. 

The processes get the coordinates and the signal strength that corresponds to the \Gls{MAC-address} of the data that the particular \Gls{GaussianProcess} was created with. For each particle the weights are computed by multiplying the computed values from each \Gls{MAC-address}. So for $n$ \Gls{MAC-address}es we get the following formula to compute the weight:
%ToDo: change notation, look up how weights in Monte Carlo are denoted, and find notation for \Gls{GaussianProcess} set.
\begin{equation}
w(x_*) = \prod_{i=1}^np(z_{t}^{[i]}|x_*)
\end{equation} 

At the end the highest particle weight determines which position is the most likely one. This position is then sent to AMCL. AMCL will use it as a new start position and will proceed the localization from those coordinates. 

AMCL will reset its own particles and spread them around that location. How far the particles are spread is determined by a covariance matrix. These values are not computed, but set manually.

So the first use case would be the manual initiation of the position estimation in order to give AMCL a seed for the localization. This would be similar to a \gls{GlobalLocalization}. The robot doesn't know where it is on the map and has to deduce its own position with the available data. 

Another use case would be in case of a localization failure. To determine a localization failure we use the ``/amcl\_failure\_probability'' published by AMCL using the 2D laser range finder.

We specify a threshold. If the computed value from equation \ref{eq:quality} is higher than the threshold, the Wi-Fi position estimation is triggered. A position will be computed and sent to AMCL as a new start position. 

\section{Using the Wi-Fi Position Estimation}
\begin{figure}[htbp]
	\centering
		\includegraphics[width=\textwidth,height=\textheight,keepaspectratio]{./Figures/laserscanner.JPG}
		\rule{35em}{0.5pt}
	\caption[Laser Range Finder]{The laser range finder used for the TurtleBot.}
	\label{fig:laserrangefinder}
\end{figure}
\begin{figure}[htbp]
	\centering
		\includegraphics[width=\textwidth,height=\textheight,keepaspectratio]{./Figures/hallway2.JPG}
		\rule{35em}{0.5pt}
	\caption[Hallway]{The hallway of the TAMS research group at the University of Hamburg where we tested the software and carried out the experiments in chapter \ref{Chapter4}.}
	\label{fig:hallway}
\end{figure}
In the preceding sections we explained the implementation of all the parts of the system. So this chapter will discuss the appliance of said software in the real world. The robot used is a TurtleBot 2 (http://www.turtlebot.com, accessed on 02.11.2016) with a Kinect \citep{ece21221}. To complement the Kinect there is also the \gls{LaserRangeFinder} ''Hokuyo UTM-30LX`` \citep{laser} installed. It can be seen in figure \ref{fig:laserrangefinder}. According to Microsoft the Kinect has a scanning range of 4 meters, but the data is still usable some meters beyond \citep{ece21221}. The \gls{LaserRangeFinder} has a scanning range of 30 meters \citep{laser}, so it is vastly superior for our purposes.  The environment we tested the software in was mainly a long hallway at the TAMS research group at the University of Hamburg that can be seen in figure \ref{fig:hallway}. 

The goal now is to put the Wi-Fi position estimation to use. The first step here is to collect data of the environment the robot is supposed to localize itself in. This can be done with AMCL from section \ref{sec:amcl} by using the laser range finder, the Wi-Fi publisher from section \ref{sec:publisher} and the Wi-Fi data collector from section \ref{sec:data_coll}. So one needs to start the nodes one after another. It is also important to make sure that AMCL's localization is close to the real position, so it is advisable to set the position manually at the start. An easy way to do this is the RViz-node. It can visualize the map used and offers the option to publish a new initial-pose to AMCL via clicking on the correct position on the map. The data collector offers different configurations. An important one is whether the Wi-Fi data should always be collected or only when standing still. In our tests we decided for the latter. To configure the node the launch file for the node can be customized. It contains the used parameters.

So when the nodes are started the Wi-Fi data can be collected by calling the rosservice ``/start\_wifi\_scan'' with the value ``true''. Now the data is recorded according to the chosen configuration. Another node not mentioned yet is the map\_traverser node. This node gets a list of $x$- and $y$-coordinates as parameters and then drives to these positions one after the other, by using the move\_base node. This node can also be used to record the data. Another method is to operate the robot via keyboard with the ``turtlebot\_teleop'' package (http://wiki.ros.org/turtlebot\_teleop, accessed on 27.10.2016). The Wi-Fi and position data is then saved in CSV-files. The Wi-Fi data can also be visualized by using occupancy grids. Calling the service ``/publish\_wifi\_maps'' publishes the occupancy grids that can be used to visualize the currently collected data. The topics the occupancy grids are published to are named after the data's \Gls{MAC-address} and an identifier signifying the nature of the visualization. There are four identifiers:
\begin{itemize}
\item ln (local normalized): normalized with the minimum and maximum of only this access point.
\item li (local interpolated): normalized with the minimum and maximum of only this access point and interpolated.
\item gn (global normalized): normalized with the minimum and maximum of all access points.
\item gi (global interpolated): normalized with the minimum and maximum of all access points and interpolated.
\end{itemize}

The collected Wi-Fi data can then be used as training data for the \Gls{GaussianProcess}. The Wi-Fi position estimation node is given a parameter that contains the path to a folder with the CSV-files for the training data. There is a launch file that lists these parameters and should be customized to point to the correct folder.

Once started the node either trains the \Gls{GaussianProcess}es created from the training data, or if this was already done when previously running the node it loads the already optimized \gls{Hyperparameter}s. The optimized \gls{Hyperparameter}s are saved as CSV-files as well, in a subfolder of the folder containing the training data. Once the node has been initialized fully it can be used to approximate the position of the robot. In order to send the position estimation to AMCL the service called ``/compute\_amcl\_start\_point'' can be used. Another parameter given to the Wi-Fi position estimation node is a threshold for the ``/amcl\_failure\_probability'' value published by AMCL. Is the value above this threshold, the Wi-Fi position estimation is triggered to recover from the localization failure. 

In this chapter we discussed the implementation of all parts needed for the Wi-Fi position estimation. At first we explained the ROS-framework used for the implementation of the system. Afterwards the parts needed to collect the Wi-Fi data were shown, with the Wi-Fi data publisher, AMCL and the Wi-Fi data collector. The collected data could then be used as training data. In order to implement the position estimation, the implementation of the Gaussian process and its application for the Wi-Fi position esitmation node were discussed. At last we gave information on the particular hardware used and explained how to use the implemented software. 

We now have a system that is able to estimate the position via Wi-Fi data. In the next chapter we will use it in different experiments to determine how effective it is for the proposed tasks. 

\chapter{Experiments and Results} % Write in your own chapter title
\label{Chapter4}
\lhead{Chapter 4. \emph{Experiments and Results}} % Write in your own chapter title to set the page header

\section{Data Analysis and Experiments}
\subsection{Wi-Fi data}
First let's examine the Wi-Fi data and look at how it behaves. 
\begin{figure}[htbp]
	\centering
		\includegraphics[width=\textwidth,height=\textheight,keepaspectratio]{./Figures/wifi_hallway.png}
		\rule{35em}{0.5pt}
	\caption[Hallway Wi-Fi data]{Visualization of Wi-Fi data collected on a hallway. The access point the signal originated from is on the same floor.}
	\label{fig:hallway_same_floor_wifi}
\end{figure}
Figure \ref{fig:hallway_same_floor_wifi} shows an example of the Wi-Fi strength of one access point. The signal reaches its peak close to the access point, marked by the yellow color. The further away the data was collected the weaker the signal was. In this example the access point was in the same room and there are pretty much no obstacles between the access point and the robot at any position. 
\begin{figure}[htbp]
	\centering
		\includegraphics[width=\textwidth,height=\textheight,keepaspectratio]{./Figures/wifi_hallway_from_different_floor.png}
		\rule{35em}{0.5pt}
	\caption[Hallway Wi-Fi data]{Visualization of Wi-Fi data collected on a hallway. In this example the access point is on a different floor.}
	\label{fig:hallway_different_floor_wifi}
\end{figure}
Turning to a different example from figure \ref{fig:hallway_different_floor_wifi}. As one can see the signal never reaches the strength of figure \ref{fig:hallway_same_floor_wifi}. But still the peak is at the center of the recorded data. To both sides then it gets weaker until it the Wi-Fi receiver can't pick it up anymore. It is reasonable to assume that the signal originates from an access point on another floor, so either from a hallway above or below. Still even though the signal is obstructed it is evenly distributed. This is probably due to the fact that no matter where the robot recorded the signal on the hallway, the signal was always obstructed by walls and thus was weakened equally. 

\begin{figure}[htbp]
	\centering
		\includegraphics[width=\textwidth,height=\textheight,keepaspectratio]{./Figures/wifi_lab.png}
		\rule{35em}{0.5pt}
	\caption[Hallway Wi-Fi data]{Visualization of Wi-Fi data collected in a room.}
	\label{fig:wifi_lab}
\end{figure}
Now figure \ref{fig:wifi_lab} paints a different picture. We can see a clear peak in the upper right corner of the room. And indeed the access point is installed there behind the wall. But the signal doesn't weaken gradually like it was on the floor. Even though in its entirety it does indeed get weaker the further away it was recorded, there are local peaks in different places. 

This is probably due to the nature of this room. There are tables, chairs and a couch, which can all lead to the obstruction of the signal. Especially because the recording robot was relatively small. 

\subsection{Wi-Fi Position Estimation accuracy}
For the Wi-Fi position estimation to be actually useful it is important that it is accurate to a certain extent. It is not necessary to achieve accuracy on the level of a few centimeters, but it can only be useful if it can reduce the possible positions to a radius of a few meters. The first experiment was designed to compare the actual position of the robot with the estimated position from the Wi-Fi data. 

To determine the actual position amcl is used. The position was determined at the start of the localization, so that it is as accurate as possible. The test environment was a long hallway. The hallway can be seen in figure \ref{fig:hallway_same_floor_wifi} This would be the kind of environment where the Wi-Fi position estimation could give an advantage over the usual approach to global localization because there aren't any characteristics that stand out and there is a lot of symmetry. 

The robot drives to different points on the map that were chosen beforehand. When it arrived the robot will execute the Wi-Fi position estimation. The result will be compared with the coordinates given by amcl. The difference between the two results will be computed.

The result of the experiment was, that on average the difference to the ground truth was 1.86 meters.
% % % % % % % % % % % % % % % % % % % % % % % % % % % % % % % % % % % % % % % % % % % % % % % % % % % % % % % % % % %
% Result here.
\subsection{Amcl with Wi-Fi Position Estimation}
\subsubsection{Global Localization}
The proposed potential use for the position estimation was that it could give a starting position for amcl. This experiment will answer in how far the position estimation can aid amcl. Once again a hallway is used. A set of goals for the robot to drive to is used. When a goal was reached the two different methods for global localization are compared.

One method uses the typical global localization where the particles are spread over the entire map and then converge to one pose. The other method will use the Wi-Fi position estimation as a seed and will lead to a much smaller area for the particles to be spread out in. 

When the estimation was performed the robot will turn 360 degrees and then drive to the next planned goal. From there the procedure is repeated. 

At the arrival at the next goal the difference to the actual position was measured. Using the Wi-Fi position estimation the mean error was 2.40 meters. With the global localization the mean error was 15.39 meters. 

So as one can see approximating the position beforehand via the Wi-Fi position estimation was a big advantage in an environment with no remarkable landmarks. The global localization was very far off from the true pose in most cases. But the Wi-Fi position estimation also had a notable difference to the ground truth. 
% % % % % % % % % % % % % % % % % % % % % % % % % % % % % % % % % % % % % %
% Results here:
\subsubsection{Kidnapped Robot Problem}
Another case where the position estimation can be used in amcl, is when the localization fails. This kind of problem is called the kidnapped robot problem. 

In this experiment the robot will drive up and down the hallway. The values from equation \ref{eq:decay} are used to compare the weights in the long term, with the weights in the short term. This means the robot needs to be localized correctly for some time, so that it can be detected that the robot was kidnapped. So at first the robot will just drive up and down the hallway, with amcl localizing the robot. Then the position of the robot in amcl, will be set to a random new position. Now the robot will be localized on the wrong position. 

Now the usual approach from amcl, explained in section \ref{sec:krp}, will be compared with the approach from the Wi-Fi position estimation, where the position estimation will be started when the value from equation \ref{eq:quality} exceeds a certain value. 

\chapter{Conclusion} % Write in your own chapter title
\label{Chapter5}
\fancyhead[LE,RO]{Chapter 5. \emph{Conclusion}}
%\lhead{Chapter 5. \emph{Conclusion}} % Write in your own chapter title to set the page header
The Wi-Fi position estimation can be realized with Gaussian processes. The result of the first experiment, where the accuracy of the Wi-Fi position estimation was tested, was a mean error of around 1.81 meters to the ground truth. This value is low enough to use the Wi-Fi position estimation to give a global localization that is using a laser range finder a rough position estimate at the start. 

In the next experiment we determined how suitable the Wi-Fi position estimation is for the use for a global localization. The result of the Wi-Fi position estimation was used as an initial position for the Monte Carlo localization with a laser range finder. The result was that on average the position differed only 2.23 meters from the ground truth. This is a big improvement over the global localization provided by AMCL, which had a mean error of 17.34 meters. In 9 out of 19 cases the global localization using the Wi-Fi position estimation was able to approximately find the true position. This too is a big improvement over the global localization from AMCL, that was only able to do this in 1 out of 19 cases. 

The experiments were carried out on a hallway and especially in the second experiment the problems laser range finders can face in such in an environment became apparent. There were no outstanding landmarks in the hallway. Many parts of the map look very similar. So when AMCL was used for the global localization it happened that the robot was not able to distinguish one end of the hallway from the other end. In other cases the robot was localizing itself in the wrong part of the hallway, because many parts of the hallway's outline are very similar. This symmetry and lack of any outstanding features in the outline were the main causes for the difficulties AMCL's global localization was facing. The global localization using the Wi-Fi position estimation was suffering from these problems as well, but to a much lesser extent. Sometimes the robot would localize itself in the wrong direction or in a part of the hallway that was a few meters away from the true position. This is a big improvement over the issues the global localization was facing when only using the laser range finder. While the experiments showed that we can almost be certain that the global localization using only the laser range finder will fail, the Wi-Fi position estimation worked almost 50\% of the time. 

The method to detect if the robot was kidnapped was working well and was able to correctly identify the problem in 10 out of 10 tries. So using this the Wi-Fi position estimation are also be able to re-localize the robot on the correct position in case of a localization failure.

In practice we foresee some drawbacks to the Wi-Fi position estimation. First of all the Wi-Fi scan to fetch the signal strengths and MAC-addresses of the access points in the environment can take a few seconds. Also the computations involved in the Wi-Fi position estimation can take some time depending on the number of access points observed, training data involved and the number of particles used. For the experiments on the hallway usually a large number of access points was observed and with 1000 particles the Wi-Fi position estimation was taking around 30 seconds to compute an approximate location. This is fine for an initial global localization, because without knowing the robot's position, moving around would be difficult anyway. In the case of a localization failure, when the Wi-Fi position estimation is triggered by the ``/amcl\_failure\_probability'', the robot would need to stop for the time of the computation of the new initial position as well. 

\chapter{Outlook} % Write in your own chapter title
\label{Chapter6}
\fancyhead[LE,RO]{Chapter 6. \emph{Outlook}}
%\lhead{Chapter 6. \emph{Outlook}} % Write in your own chapter title to set the page header 

The Wi-Fi position estimation precision is good enough in the experiment with a mean error of 1.81 meters, so that it can be used for the global localization for AMCL with a laser range finder. We used a radial basis function kernel for the Gaussian processes, but we did not investigate any other kernels. Investigating different kernels could potentially improve the accuracy even further.
 
The Wi-Fi signals can only be used to infer information about the position, but not about the orientation of the robot. One could for example use a compass to get this kind of information. Right now the particles that are spread by the Wi-Fi position estimation have a random orientation. Using a compass one could infer an approximate orientation and thus would be able to spread particles that have that orientation. When carrying out the second experiment where the Wi-Fi position estimation was used for the global localization we observed that the robot would often be localized in the wrong direction on the hallway. So using a compass could clearly lead to an improvement in this case. 

Another concern is the computation time of the Wi-Fi position estimation. When the global localization using the Wi-Fi position estimation is executed, random positions on the map are generated that are used for the particles. So every Gaussian process for each of the networks has to compute a Gaussian distribution for all those particles. Depending on the number of particles, size of the training data and the number of observed Wi-Fi networks in the moment, this can cause long computation times. One solution would be to pre-compute the Gaussian distributions when the Wi-Fi position estimation node is initialized. So the random positions are all generated beforehand and when the Wi-Fi position estimation is actually called these randomly generated positions are used. The obvious consequence here is that in case we only generate the random positions once, the position of the particles would not change in subsequent executions. A solution would be to draw a new set of random positions after a Wi-Fi position estimation was performed and calculate the new Gaussian distributions then. 

Right now the robot has to stand whenever the Wi-Fi position estimation is executed. The reason for this is the already mentioned computation time, but another reason is also the time the Wi-Fi scan takes, which usually is a few seconds. The different Wi-Fi channels are scanned one after the other. When the robot moves in the few seconds that the scan is taking, then the data from the different channels is spread over the path the robot moved on. So we do not know the actual position the specific networks were actually recorded on. But as we explained it is also possible to scan the different Wi-Fi channels separately. This is a lot faster than the whole scan, but we will also fetch less Wi-Fi networks. This makes some optimizations possible. First off we would be able to scan the different channels one after the other, but get the data from each channel instantly instead of waiting for scanning all the other channels. This would give us the ability to process the data when it is only a fraction of a second old. Another possible optimization would be to only scan a selection of Wi-Fi channels. The command used for the Wi-Fi publisher also provides information about the channel used for a network. So when recording the training data, we would be able to save those channels, and when we use the Wi-Fi position estimation we would only need to scan these particular channels. This could also help with the computing time. When we scan the channels individually we observe less access points and therefore less computation time is needed for the Gaussian processes. 

This can also help to implement the Wi-Fi sensor model for direct use with AMCL. The faster we are able to update the weights with new data, the better. We could apply the computed weights directly to the particles from AMCL. This can be done by multiplying the weights computed with the measurement model for the laser range finder with the weights computed by the Wi-Fi receiver sensor measurement model. One problem here will be that the weights produced by the Wi-Fi sensor model can differ from iteration to iteration. Their value is dependent on the number of Wi-Fi access points observed, so they would have to be normalized somehow before used with AMCL. If the channels are scanned individually the data can be used instantly to compute the weight, instead of waiting a few seconds for the scan of all channels to end.  Another factor for the computation time is the number of particles. AMCL varies the number of particles. When they are clustered around one position there are less particles compared to the start of a global localization when the particles are spread over the entire map. Less particles would lessen the computation time as well. 

Another factor for the computation time is the size of the training data set. If we would find a strategy that is able to reduce that set while keeping the accuracy of the Wi-Fi position estimation and the quality of the results from the Wi-Fi sensor model relatively stable, then this would also help to reduce the computation time.

We used the weights computed by AMCL using the laser range finder for the detection of the kidnapped robot problem. One possible way to use the Wi-Fi position estimation for this task would be to compare a position computed by it with a position computed by AMCL. Is the difference larger than a certain threshold, then AMCL is re-initialized using the Wi-Fi position estimation. 

%\input{./Chapters/Chapter2} % Background Theory 

%\input{./Chapters/Chapter3} % Experimental Setup

%\input{./Chapters/Chapter4} % Experiment 1

%\input{./Chapters/Chapter5} % Experiment 2

%\input{./Chapters/Chapter6} % Results and Discussion

%\input{./Chapters/Chapter7} % Conclusion

%% ----------------------------------------------------------------
% Now begin the Appendices, including them as separate files

\addtocontents{toc}{\vspace{2em}} % Add a gap in the Contents, for aesthetics

\appendix % Cue to tell LaTeX that the following 'chapters' are Appendices

\input{./Appendices/AppendixA}	% Appendix Title

%\input{./Appendices/AppendixB} % Appendix Title

%\input{./Appendices/AppendixC} % Appendix Title

\addtocontents{toc}{\vspace{2em}}  % Add a gap in the Contents, for aesthetics
\backmatter

%% ----------------------------------------------------------------
\label{Bibliography}
\lhead{\emph{Bibliography}}  % Change the left side page header to "Bibliography"
%\bibliographystyle{apalike}  % Use the "unsrtnat" BibTeX style for formatting the Bibliography
%\bibliographystyle{dcu}
\bibliographystyle{apalike}
\bibliography{Bibliography}  % The references (bibliography) information are stored in the file named "Bibliography.bib"

\includepdf[pages=-, offset=75 -75]{./lastpage.pdf}

\end{document}  % The End
%% ----------------------------------------------------------------