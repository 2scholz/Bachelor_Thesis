\chapter{Introduction} % Write in your own chapter title
\label{Chapter1}
\lhead{Chapter 1. \emph{Introduction}} % Write in your own chapter title to set the page header

\section{Motivation}
Localization is an important problem in robotics. It is needed in order to solve a wide range of problems. Even simple tasks in households require a robot to be able to determine its own position in the environment. It needs to be able to get from point A to point B. Being able to do this, a robot could for example bring a fresh cup of coffee and return the used cup back to the kitchen. It can help with cleaning tasks\citep{pinheiro2015cleaning}. It can also be highly important in industrial environments. Hamburg has one of the world's largest and busiest ports. In order to bring the shipping containers from one place to the other autonomous transport vehicles can be used. These are just a few of many real world examples, where localization is used.

Another modern example of using localization is the autonomous vehicle. The Global Positioning System(\Gls{GPS})\citep{misra2006global} can give us a good idea about the position of the car. But \Gls{GPS} is too inaccurate for the high demands of autonomous vehicles. \citet{DBLP:conf/rss/LevinsonMT07} for example used Light Detection and Ranging(\Gls{LIDAR})\citep{Lidar} additionally to the \Gls{GPS} to vastly improve the localizations accuracy. The \Gls{LIDAR} is a vision based sensor that is similar to a radar. It can be used to map out nearby obstacles. The \Gls{LIDAR} is not well suited to estimate the position on a large map with no prior information, so the \Gls{GPS} can be used to provide a rough position estimate. On the other hand the \Gls{GPS} is inaccurate, so once a rough position is estimated, the \Gls{LIDAR} can help to narrow the position down. 

In robotics another similar sensor to the \Gls{LIDAR}, a 2D \gls{LaserRangeFinder} is often used for localization. With the \gls{LaserRangeFinder} the robot is also able to detect the distance of obstacles near itself. There are many situations where a robot can not rely on \Gls{GPS}. In many buildings its signal is obstructed.
Localizing without any prior information about the robot's position is called \gls{GlobalLocalization}.
There are existing approaches to do this with just a vision-based sensor like the \Gls{LIDAR} or a \gls{LaserRangeFinder}. Usually when localizing only a small part of the map is taken into account, because we have a rough idea about the position. When using \gls{GlobalLocalization} with only a \gls{LaserRangeFinder} the whole map is taken into account at the beginning. 
But these methods have the downside that they are often unreliable and lead to localization failures that are hard to recover from. This is especially true for environments that don't have many unique characteristics. When localizing with a \gls{LaserRangeFinder} one usually creates a map with the existing obstacles and the outline of the room. Often many parts of these maps look very similar. The less distinctive features the map has, the more likely it is that the \gls{GlobalLocalization} with a \gls{LaserRangeFinder} fails. 
Many places indoors can cause those problems. Many rooms are symmetric, so it can be hard to differentiate between the walls. Many buildings have rooms with the same outline, like for example many office-environments. Office environments can also be quite large, and the larger the space that has to be taken into account the more likely it is, that a \gls{GlobalLocalization} only using a \gls{LaserRangeFinder} will fail. So this can cause problems especially in buildings.
 
In most buildings there are a lot of different Wi-Fi\citep{ieee802.11-2012} networks. The networks \Gls{MAC-address} provides a unique identifier for each signal. The signal strengths can be a good indicator about how far the wireless access point is away. And while one signal alone wouldn't be enough to infer the position multiple signals from different access points can be used to do just that.

This could be used to enhance the localization with laser scanners. When at the start no position estimation is given it could be computed with the Wi-Fi data. Also in case of a localization failure the Wi-Fi signals could be used to infer the correct position.

This combines the strengths of both the position estimation with Wi-Fi signals and the localization with a laser scanner. The Wi-Fi signals have a unique identifier and will only be observed in a certain range on a map. The outline of a wall could be nearly identical in different places, but overall the localization with a laser scanner is very accurate once the true position was determined. 

In Chapter 2 we will explain the basics of probability based localization, with the so called \Gls{MonteCarlo}\citep{Dellaert_1999_533} as an example. 
Furthermore we will explore how a Wi-Fi receiver can be used with the \Gls{MonteCarlo} and explain how \Gls{GaussianProcess}\citep{Rasmussen:2005:GPM:1162254} regression can be used for this purpose.
Chapter 3 details the implementation of a Wi-Fi position estimation using the knowledge from Chapter 2 about \Gls{MonteCarlo} and \Gls{GaussianProcess} regression.
In Chapter 4 experiments and results regarding the usage of the Wi-Fi position estimation will be discussed.
In chapter 5 we will reflect the results and come to a conclusion. 
Lastly chapter 6 provides an outlook for future applications and possibilities.

\section{Related Work}
One of the most popular methods for localization is the \Gls{MonteCarlo}. We are going to use this method, both for localization via a 2D range finder, and parts of it also to estimate the position via Wi-Fi. \citet{Thrun:2005:PR:1121596} give an extensive overview over the \Gls{MonteCarlo}.

In order for the \Gls{MonteCarlo} to work, one needs a measurement model for the sensor that is used. \citet{Thrun:2005:PR:1121596} provide models that can be used for a 2D range finder. In order to estimate the position with a Wi-Fi receiver, one needs to implement a different measurement model.

There have been different approaches to solving the localization problem with Wi-Fi signals. \citet{serrano2012robot} used a propagation model to estimate the signal strength at a given point on the map. The advantage of this technique is that only the positions of the access points have to be known. But it is a lot more inaccurate than other approaches. The problem is that inside a building the Wi-Fi signal is often obstructed by obstacles, like walls, furniture or humans.  This makes the propagation model very inaccurate and thus also leads to inaccurate position estimations.

A different approach to the problem uses a map that was created beforehand. \citet{biswas2010wifi} recorded Wi-Fi signals a in grid like fashion, recording in a certain interval. In order to interpolate the Wi-Fi signal from the recordings, linear interpolation was used. The drawback here is that the Wi-Fi signals have to be recorded at certain positions on the map. This makes the process of creating the map more complicated and time consuming.

In order to create something akin to a map from recorded Wi-Fi data, using regression proved to be useful. As it turns out \Glspl{GaussianProcess}es are well suited for this kind of problem. \citet{Rasmussen:2005:GPM:1162254} give an extensive overview of how they can be used in machine learning. 

\citet{ferris2006gaussian} use \Gls{GaussianProcess}es for building the map and for localization. This means the signal strength can be recorded at random places. Thus the needed data for creating the map can be recorded while the robot drives around or even does other tasks. 
In our approach we want to determine whether such a method can be used for \gls{GlobalLocalization} in an office environment with a 2D \gls{LaserRangeFinder} and a wheeled robot.
There are already existing investigations into a \gls{GlobalLocalization} with Wi-Fi data, but in different contexts. \citet{duvallet2008wifi} used it in an industrial environment to enhance the previously developed laser beacon localization\citep{4209248}. The most striking difference here is the environment, which is large and spacious compared to the often cramped spaces used as offices. Also the number of Wi-Fi signals that can be observed in office spaces is usually a lot higher than in industrial environments.
\citet{DBLP:conf/icra/ItoEKTSB14} used Wi-Fi for \gls{GlobalLocalization} for an iPhone 4 with an \Gls{RGBD}, so a sensor like the Microsoft Kinect for example that can sense colors and depth. For the \Gls{RGBD} they used a map created from the floor-plan, that is less accurate then the maps usually used. Here an obvious disparity to our test environment is the platform used. While the phone is used in conjunction with an \Gls{RGBD}, which is comparable to a 2D \gls{LaserRangeFinder}, the phone does not have any wheels that can provide odometry. Here they resort to an alternative for the odometry method using the \Gls{RGBD}.  