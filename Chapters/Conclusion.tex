\chapter{Conclusion} % Write in your own chapter title
\label{Chapter5}
\lhead{Chapter 5. \emph{Conclusion}} % Write in your own chapter title to set the page header
The Wi-Fi position estimation can be realized with Gaussian processes. The results of the first experiment, where the accuracy of the Wi-Fi position estimation was tested, was a mean error of around 1.81 meters to the ground truth. This value is low enough to use the Wi-Fi position estimation for the global localization. It is able to provide a rough estimate of the location beforehand. 

In the next experiment we determined how suitable the Wi-Fi position estimation is for the use as global localization. The result of the Wi-Fi position estimation was used as an initial position for the Monte Carlo localization with a laser range finder. The result was that on average the position differed only 2.23 meters from the ground truth. This is a big improvement over the global localization provided by AMCL, which had a mean error of 17.34 meters. In 9 out of 19 cases the global localization using the Wi-Fi position estimation was able to approximately find the true position. This too is a big improvement over the global localization from AMCL, that was only able to do this in 1 out of 19 cases. 

The experiments were carried out on a hallway and especially in the second experiment the problems laser range finders can face in such in an environment became apparent. There were no outstanding landmarks in the hallway. Many parts of the map look very similar. So when AMCL was using the global localization it could happen that the robot was not able to distinguish one end of the hallway from the other end. In other cases the robot was localizing itself in the wrong part of the hallway, because many parts of the hallway's outline are very similar as well. This symmetry and lack of any outstanding features in the outline were the main causes for the difficulties AMCL's global localization was facing. The global localization using the Wi-Fi position estimation was suffering from these problems as well, but to a lesser extent. So it happened that the robot would localize itself in the wrong direction or in a part of the hallway that was a few meters away from the true position. But the Wi-Fi position estimation was able to reduce the issues caused by those circumstances.

The method to detect if the robot was kidnapped was working well and was able to correctly identify the problem in 10 out of 10 tries. So using this the Wi-Fi position estimation would also be able to help to re-localize the robot on the correct position. 

In practice we foresee some drawbacks to the Wi-Fi position estimation. First of all the Wi-Fi scan to fetch the signal strengths and MAC-addresses of the access points in the environment can take a few seconds. Then the computations involved in the Wi-Fi position estimation can take a few seconds depending on the number of access points observed, training data involved and the number of particles used. For the experiments on the hallway usually a large number of access points was observed and with 1000 particles the Wi-Fi position estimation was taking 30 seconds to compute an approximate location. This is fine for an initial global localization, because the robot without a localization would not know where to go anyway. In the case of a localization failure, when the Wi-Fi position estimation is triggered by the ``/amcl\_failure\_probability'', the robot would need to stop for the time of the computation of the new initial position as well. 