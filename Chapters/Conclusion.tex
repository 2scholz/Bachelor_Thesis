\chapter{Conclusion} % Write in your own chapter title
\label{Chapter5}
\lhead{Chapter 5. \emph{Conclusion}} % Write in your own chapter title to set the page header

The Wi-Fi position estimation can be realized with \Gls{GaussianProcess}es. The results of the first experiment had a mean error of around 1.81 meters to the ground truth. As explained the position is computed by spreading particles over the map and then choosing the one with the highest weight. 
The Wi-Fi position estimation can be helpful for the \Gls{MonteCarlo} with a \gls{LaserRangeFinder}. The experiment showed that in the case of \gls{GlobalLocalization} the Wi-Fi position estimation can lead to a more accurate position estimation. With the Wi-Fi position estimation the error to the ground truth after driving for a few seconds was 3.39 meters, while it was 17.34 meters for the \gls{GlobalLocalization} of amcl. 

But of course there are drawbacks to the Wi-Fi position estimation. For one scanning for new data takes a few seconds. That means when the robot drives around the newest data for an access point can already be seconds old and thus be already meters away from the current position. Another problem is the computation time for the \Gls{GaussianProcess}es. When using the Wi-Fi position estimation for \gls{GlobalLocalization} it can take a few seconds as well to get a result. The factors here are the number of access points available and the numbers of particles used. To use it for the \gls{GlobalLocalization} a large number of particles are needed. Of course this also depends on the size of the map. But once again when the robot drives around it can happen that the result of the Wi-Fi position estimation is already based on data that is tens of seconds old and thus based on a different position on the map. 

These are limitations that will pose problems when the Wi-Fi position estimation is actually to be used in real world applications. As for the \gls{GlobalLocalization} it can still be usable. The robot needs to know its own position before driving around and the Wi-Fi position estimation can help with that. So it will not be a problem when the robot has to stand while computing it. In the case of a localization failure, so the kidnapped robot problem, the robot is usually already doing a task, so it is probable that the robot will drive around and as discussed this can pose a problem for the position estimation. So when the a localization failure is detected the robot would need to stop its task and stand still until a position was computed and continue afterwards. 

Another problem can arise when the Wi-Fi position estimation computes a position that is too far off from the true position. Since the detection for the localization failure is based on the difference between the short- and long-term average weights it likely that this failure will not be recognized.