\chapter{Conclusion} % Write in your own chapter title
\label{Chapter5}
\lhead{Chapter 5. \emph{Conclusion}} % Write in your own chapter title to set the page header

\section{Conclusion}
The Wi-Fi position estimation can be realized with Gaussian processes. The results of the first experiment had a mean error of around 1.86 meters to the ground truth. As explained the position is computed by spreading particles over the map and then choosing the one with the highest weight. 
The Wi-Fi position estimation can be helpful for the Monte Carlo localization with a laser scanner. The experiment showed that in the case of global localization the Wi-Fi position estimation can lead to a more accurate position estimation. With the Wi-Fi position estimation the error to the ground truth after driving for a few seconds was 2.40 meters, while it was 15.39 meters for the global localization of amcl. 

But of course there are drawbacks to the Wi-Fi position estimation. For one scanning for new data takes a few seconds. That means when the robot drives around the newest data for an access point can already be seconds old and thus be already meters away from the current position. Another problem is the computation time for the Gaussian processes. When using the Wi-Fi position estimation for global localization it can take a few seconds as well to get a result. The factors here are the number of access points available and the numbers of particles used. To use it for the global localization a large number of particles are needed. Of course this also depends on the size of the map. But once again when the robot drives around it can happen that the result of the Wi-Fi position estimation is already based on data that is tens of seconds old and thus based on a different position on the map. 

These are limitations that will pose problems when the Wi-Fi position estimation is actually to be used in real world applications. As for the global localization it can still be usable. The robot needs to know its own position before driving around and the Wi-Fi position estimation can help with that. So it will not be a problem when the robot has to stand while computing it. In the case of a localization failure, so the kidnapped robot problem, the robot is usually already doing a task, so it is probable that the robot will drive around and as discussed this can pose a problem for the position estimation. So when the a localization failure is detected the robot would need to stop its task and stand still until a position was computed and continue afterwards. 

Another problem can arise when the Wi-Fi position estimation computes a position that is too far off from the true position. Since the detection for the localization failure is based on the difference between the short- and long-term average weights it likely that this failure will not be recognized.

\section{Outlook}
Thus far only a position estimation was realized. While this can be helpful for the Monte Carlo localization with a laser scanner, it was shown that there are still a lot of problematic limitations. A sensor measurement model for the Wi-Fi receiver was implemented. This could be used directly in amcl. For the Monte Carlo localization it is no problem to incorporate different sensors. The weights can simply be computed by computing the product of all sensor measurement models. But once again the biggest obstacle here is the time requirement for both the Wi-Fi scan and the computation time of the Wi-Fi measurement model. As for the Wi-Fi measurement model this problem could be eased by the fact that amcl uses a lot less particles when it converged to a position. Another trick that could be used is to omit some access points per iteration. But even then the time a Wi-Fi scan takes would pose a problem. Also the question if incorporating the Wi-Fi data into the localization would lead to a better localization still stands. With the measurement model used here for the Wi-Fi receiver it is not possible to determine the direction of the robot, but only its position. The localization with a laser scanner is already reliable once the true pose was found, so the contribution of Wi-Fi data to this problem remains questionable. 

Similar concerns stand for a localization realized purely with the Wi-Fi measurement model, without a laser scanner. Once again the time a Wi-Fi scan takes and the computation time of the Wi-Fi sensor model are too high. 

As for the use of the Wi-Fi position estimation for the global localization, we still see possibilities for optimizations. Right now particles are spread over the entire map. So with a bigger map one either heightens the particle count, or one accepts the worse position estimation. But with the recorded Wi-Fi signals beforehand and the current recorded Wi-Fi signals one can approximate the area, where these recordings take place. So by looking at the MAC-addresses observed and then looking at the area where these particular MAC-addresses were observed one can at least do an approximation of the area where the robot is and then only spread the particles there. Of course this only works if one assumes that the training data covers the area where the robot is positioned, but in most cases that should be given. 