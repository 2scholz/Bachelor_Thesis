\chapter{Outlook} % Write in your own chapter title
\label{Chapter6}
\lhead{Chapter 6. \emph{Outlook}} % Write in your own chapter title to set the page header 

The Wi-Fi position estimation precision is with a mean error of 1.81 meters in the experiment good enough, so that it can be used for the global localization for AMCL with a laser range finder. The Wi-Fi signals can only be used to infer information about the position, but not about the orientation of the robot. One could for example use a compass to get this kind of information. Right now the particles, that are spread by the Wi-Fi position estimation have a random orientation. Using a compass one could infer an approximate orientation and thus would be able to spread particles that have that orientation. When carrying out the second experiment where the Wi-Fi position estimation was used for the global localization we observed that the robot would often be localized in the wrong direction on the hallway. So using a compass could clearly lead to an improvement in this case. 

Another concern is the computation time of the Wi-Fi position estimation. When the global localization using the Wi-Fi position estimation is executed random positions on the map are generated that are used as particles. So every Gaussian process for each of the networks has to compute a Gaussian distribution for all those particles. Depending on the number of particles, size of the training data and the number of observed Wi-Fi networks in the moment, this can cause long computation times. One solution would be to pre-compute the Gaussian distributions when the Wi-Fi position estimation node is initialized. So the random positions are all generated beforehand and when the Wi-Fi position estimation is actually called these randomly generated positions are used. The obvious consequence here is that in case we only generate the random positions once, the position of the particles would not change in subsequent executions. A solution would be to draw a new set of random positions after a Wi-Fi position estimation was performed and calculate the new Gaussian distributions then. 

Right now the robot has to stand whenever the Wi-Fi position estimation is executed. The reason for this is the already mentioned computation time, but another reason is also the time the Wi-Fi scan takes. A full scan usually takes a few seconds. The different Wi-Fi channels are scanned one after the other. When the robot moves in the few seconds that the scan is taking, then the data from the different channels is spread over the path the robot moved on. So we do not know the actual position the specific networks where actually recorded on. But as we explained it is also possible to scan the different Wi-Fi channels separately. This is a lot faster then the whole scan, but we will also fetch less Wi-Fi networks. This makes some optimizations possible. First off we would be able to scan the different channels one after the other, but get the data from each channel instantly instead of waiting for scanning all the other channels. This would give us the ability to process the data when it is only a fraction of a second old. Another possible optimization would be to only scan a selection of Wi-Fi channels. The command used for the Wi-Fi publisher also carries information about the channel used for a network. So we would only need to scan those particular channels.

This can also help to implement the Wi-Fi sensor model directly into AMCL. So then the computed weights could be directly applied to the particles from AMCL. This can just be done by multiplying the weights computed with the measurement model for the laser range finder with the weights computed by the Wi-Fi receiver sensor measurement model. One problem here will be that the weights produced by the Wi-Fi sensor model can differ from iteration to iteration. Their value is dependent on the number of Wi-Fi access points observed. So they would have to be normalized somehow before used with AMCL. When the channels are scanned individually the data can be used instantly to compute the weight, instead of waiting a few seconds for the scan of all channels to end. This could also help with the computing time. When we scan the channels individually we observe less access points and therefore less computation time is needed for the Gaussian processes. Another factor is the number of particles. AMCL varies the number of particles. When they are clustered around one position there are less particles compared to the start of a global localization when the particles are spread over the entire map. Less particles would lessen the computation time as well. 

Another factor for the computation time is the size of the training data set. If we would find a strategy that would be able to reduce that set while keeping the accuracy of the Wi-Fi position estimation and the quality of the results from the Wi-Fi sensor model relatively stable, then this would also help to reduce the computation time.

We used the weights computed by AMCL using the laser range finder for the detection of the kidnapped robot problem. One possible way to use the Wi-Fi position estimation for the detection would be to approximate the position using it and when the position computed by AMCL is further away than a certain threshold then AMCL is re-initialized using the Wi-Fi position estimation. 