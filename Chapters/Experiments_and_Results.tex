\chapter{Experiments and Results} % Write in your own chapter title
\label{Chapter4}
\lhead{Chapter 4. \emph{Experiments and Results}} % Write in your own chapter title to set the page header

\section{Data Analysis and Experiments}
\subsection{Wi-Fi data}
First let's examine the Wi-Fi data and look at how it behaves. 
\begin{figure}[htbp]
	\centering
		\includegraphics[width=\textwidth,height=\textheight,keepaspectratio]{./Figures/wifi_hallway.png}
		\rule{35em}{0.5pt}
	\caption[Hallway Wi-Fi data]{Visualization of Wi-Fi data collected on a hallway. The access point the signal originated from is on the same floor.}
	\label{fig:hallway_same_floor_wifi}
\end{figure}
Figure \ref{fig:hallway_same_floor_wifi} shows an example of the Wi-Fi strength of one access point. The signal reaches its peak close to the access point, marked by the yellow color. The further away the data was collected the weaker the signal was. In this example the access point was in the same room and there are pretty much no obstacles between the access point and the robot at any position. 
\begin{figure}[htbp]
	\centering
		\includegraphics[width=\textwidth,height=\textheight,keepaspectratio]{./Figures/wifi_hallway_from_different_floor.png}
		\rule{35em}{0.5pt}
	\caption[Hallway Wi-Fi data]{Visualization of Wi-Fi data collected on a hallway. In this example the access point is on a different floor.}
	\label{fig:hallway_different_floor_wifi}
\end{figure}
Turning to a different example from figure \ref{fig:hallway_different_floor_wifi}. As one can see the signal never reaches the strength of figure \ref{fig:hallway_same_floor_wifi}. But still the peak is at the center of the recorded data. To both sides then it gets weaker until it the Wi-Fi receiver can't pick it up anymore. It is reasonable to assume that the signal originates from an access point on another floor, so either from a hallway above or below. Still even though the signal is obstructed it is evenly distributed. This is probably due to the fact that no matter where the robot recorded the signal on the hallway, the signal was always obstructed by walls and thus was weakened equally. 

\begin{figure}[htbp]
	\centering
		\includegraphics[width=\textwidth,height=\textheight,keepaspectratio]{./Figures/wifi_lab.png}
		\rule{35em}{0.5pt}
	\caption[Hallway Wi-Fi data]{Visualization of Wi-Fi data collected in a room.}
	\label{fig:wifi_lab}
\end{figure}
Now figure \ref{fig:wifi_lab} paints a different picture. We can see a clear peak in the upper right corner of the room. And indeed the access point is installed there behind the wall. But the signal doesn't weaken gradually like it was on the floor. Even though in its entirety it does indeed get weaker the further away it was recorded, there are local peaks in different places. 

This is probably due to the nature of this room. There are tables, chairs and a couch, which can all lead to the obstruction of the signal. Especially because the recording robot was relatively small. 

\subsection{Wi-Fi Position Estimation accuracy}
For the Wi-Fi position estimation to be actually useful it is important that it is accurate to a certain extent. It is not necessary to achieve accuracy on the level of a few centimeters, but it can only be useful if it can reduce the possible positions to a radius of a few meters. The first experiment was designed to compare the actual position of the robot with the estimated position from the Wi-Fi data. 

To determine the actual position amcl is used. The position was determined at the start of the localization, so that it is as accurate as possible. The test environment was a long hallway. The hallway can be seen in figure \ref{fig:hallway_same_floor_wifi} This would be the kind of environment where the Wi-Fi position estimation could give an advantage over the usual approach to global localization because there aren't any characteristics that stand out and there is a lot of symmetry. 

The robot drives to different points on the map that were chosen beforehand. When it arrived the robot will execute the Wi-Fi position estimation. The result will be compared with the coordinates given by amcl. The difference between the two results will be computed.

The result of the experiment was, that on average the difference to the ground truth was 1.86 meters.
% % % % % % % % % % % % % % % % % % % % % % % % % % % % % % % % % % % % % % % % % % % % % % % % % % % % % % % % % % %
% Result here.
\subsection{Amcl with Wi-Fi Position Estimation}
\subsubsection{Global Localization}
The proposed potential use for the position estimation was that it could give a starting position for amcl. This experiment will answer in how far the position estimation can aid amcl. Once again a hallway is used. A set of goals for the robot to drive to is used. When a goal was reached the two different methods for global localization are compared.

One method uses the typical global localization where the particles are spread over the entire map and then converge to one pose. The other method will use the Wi-Fi position estimation as a seed and will lead to a much smaller area for the particles to be spread out in. 

When the estimation was performed the robot will turn 360 degrees and then drive to the next planned goal. From there the procedure is repeated. 

At the arrival at the next goal the difference to the actual position was measured. Using the Wi-Fi position estimation the mean error was 2.40 meters. With the global localization the mean error was 15.39 meters. 

So as one can see approximating the position beforehand via the Wi-Fi position estimation was a big advantage in an environment with no remarkable landmarks. The global localization was very far off from the true pose in most cases. But the Wi-Fi position estimation also had a notable difference to the ground truth. 
% % % % % % % % % % % % % % % % % % % % % % % % % % % % % % % % % % % % % %
% Results here:
\subsubsection{Kidnapped Robot Problem}
Another case where the position estimation can be used in amcl, is when the localization fails. This kind of problem is called the kidnapped robot problem. 

In this experiment the robot will drive up and down the hallway. The values from equation \ref{eq:decay} are used to compare the weights in the long term, with the weights in the short term. This means the robot needs to be localized correctly for some time, so that it can be detected that the robot was kidnapped. So at first the robot will just drive up and down the hallway, with amcl localizing the robot. Then the position of the robot in amcl, will be set to a random new position. Now the robot will be localized on the wrong position. 

Now the usual approach from amcl, explained in section \ref{sec:krp}, will be compared with the approach from the Wi-Fi position estimation, where the position estimation will be started when the value from equation \ref{eq:quality} exceeds a certain value. 